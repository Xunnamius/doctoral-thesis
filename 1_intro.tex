\chapter{Introduction} \label{chp:introduction}

\section{Thesis Statement}

With this research into filesystem, device driver, and hardware Flash
Translation Layer (FTL) based Full Disk Encryption (FDE) schemes, we primarily
consider: (1) the established wisdom in the crypto community that stream ciphers
are unsuitable for FDE and (2) exploring the tradeoff space made between total
energy use, filesystem performance, and reasonable security guarantees when
comparing specific cipher configurations. In the first case, we develop and
implement a secure approach to FDE based on stream ciphers, the proliferation of
secure hardware, and the characteristics of Log-Structured Filesystems (LFS). In
the second case, we implement a dozen stream ciphers---each exposing several
knobs---and demonstrate navigating our tradeoff space via runtime cipher
switching while maintaining reasonable security guarantees. We then present a
formal analysis of our system's security guarantees.

\section{Problem Description}

Full-drive encryption (FDE)\footnote{The common term is full-\emph{disk}
encryption, but this work targets SSDs, so we use \emph{drive}.} is an essential
technique for protecting the privacy of data at rest. For mobile devices,
maintaining data privacy is especially important as these devices contain
sensitive personal and financial data yet are easily lost or stolen. The current
standard for securing data at rest is to use the AES cipher in XTS
mode~\cite{XTS, NISTXTS}. Unfortunately, employing AES-XTS increases read/write
latency by 3--5$\times$ compared to unencrypted storage.

It is well known that authenticated encryption using \emph{stream}
ciphers---such as ChaCha20~\cite{ChaCha20}---is faster than using AES. Indeed,
Google made the case for stream ciphers over AES, switching HTTPS connections on
Chrome for Android to use a stream cipher for better
performance~\cite{google-blog}. Stream ciphers are not used for FDE, however,
for two reasons: (1) confidentiality and (2) performance. First, when applied
naively to stored data, stream ciphers are trivially vulnerable to
attacks---including \emph{many-time pad and rollback attacks}---that reveal the
plaintext by overwriting a secure storage location with the same key. Second, it
has been assumed that adding the meta-data required to resist these attacks
would ruin the stream cipher's performance advantage. Thus, the conventional
wisdom is that FDE necessarily incurs the overhead of AES-XTS or a similar
primitive.

We argue that two technological shifts in mobile device hardware overturn this
conventional wisdom, enabling confidential, high-performance storage with stream
ciphers. First, these devices commonly employ solid-state storage with Flash
Translation Layers (FTL), which operate similarly to Log-structured File Systems
(LFS)~\cite{LFS,F2FS,NILFS}. Second, mobile devices now support trusted
hardware, such as Trusted Execution Environments (TEE)~\cite{TEE,TrustZone} and
secure storage areas~\cite{eMMC-standard}. FTLs and LFSes are used to limit
sector/cell overwrites, hence extending the life of the drive. Most writes
simply appended to a log, reducing the occurrence of overwrites and the chance
for attacks. The presence of secure hardware means that drive encryption modules
have access to persistent, monotonically increasing counters that can be used to
prevent rollback attacks when overwrites do occur.

Given these trends, we propose StrongBox, a new method for securing data at
rest. StrongBox is a drop-in replacement for AES-XTS-backed FDE such as
dm-crypt~\cite{dmcrypt}; i.e. it requires no interface changes. The primary
challenge is that even with a FTL or LFS running above an SSD, filesystem blocks
will occasionally be overwritten; e.g. by segment cleaning or \emph{garbage
collection}. StrongBox overcomes this challenge by using a fast stream cipher
for confidentiality and performance with integrity preserving Message
Authentication Codes~\cite{MAC} or ``MAC tags'' and a secure, persistent
hardware counter to ensure integrity and prevent attacks. \emph{StrongBox's main
contribution is a system design enabling the first confidential,
high-performance drive encryption based on a stream cipher.}


\section{Contributions}

\TODO{List them.}
