% ---- ETD Document Class and Useful Packages ---- %
\documentclass{ucetd}
\usepackage{amsfonts}
\usepackage{amsthm}

\usepackage{epsfig, minted}
%\usepackage[utf8x]{inputenc}
\usepackage{algorithm}
\usepackage{amsmath, amssymb}
\usepackage[noend]{algpseudocode}
\usepackage{enumitem}      % adjust spacing in enums
%\usepackage{subfig}
\usepackage{caption}
\usepackage{subcaption}
\usepackage{multirow}
\usepackage{rotating}
\usepackage{wrapfig}

%% Use these commands to set biographic information for the title page:
\title{Capitalizing on Security, Performance, and Energy Tradeoffs in Full Drive Encryption Schemes for Fun and Profit}
\author{Bernard Dickens III}
\department{Computer Science}
\division{Physical Sciences}
\degree{Doctor of Philosophy}
\date{July 2020}

%% Use these commands to set a dedication and epigraph text
\dedication{Dedication Text}
\epigraph{Epigraph Text}

%% Custom additions
\usepackage[natbib=true,backend=bibtex,firstinits=true,style=numeric-comp,sorting=nyt,defernumbers,maxnames=99,maxcitenames=99]{biblatex}
\usepackage{balance}
\usepackage{adjustbox}

\addbibresource{refs.bib}

\usepackage{pgfplots}
% options for pgfplots
\pgfplotsset{compat=1.8,compat/show suggested version=false}
\usetikzlibrary{plotmarks}
\usetikzlibrary{calc}
%\pgfplotsset{compat=newest}
\pgfplotsset{
   /pgfplots/bar  cycle  list/.style={/pgfplots/cycle  list={%
        {black,fill=black!30!white,mark=none},%
        {black,fill=red!30!white,mark=none},%
        {black,fill=green!30!white,mark=none},%
        {black,fill=yellow!30!white,mark=none},%
        {black,fill=brown!30!white,mark=none},%
     }
   },
}
% begin of externalization
\usetikzlibrary{external}
\tikzexternalize[prefix=out/]
\tikzexternalize
% don't externalize todonotes
%\makeatletter
%\renewcommand{\todo}[2][]{\tikzexternaldisable\@todo[#1]{#2}\tikzexternalenable}
%\makeatother
% end of externalization
\usetikzlibrary{patterns}
\usepgfplotslibrary{groupplots}
\pgfplotsset{
every axis label/.append style={font=\small},
tick label style={font=\small},
}

\graphicspath{{./figs/}}
\date{}

\algnewcommand{\LineComment}[1]{\(\triangleright\) #1}

% some useful shortcuts
\newcommand{\ie}{\textit{i.e., }}
\newcommand{\eg}{\textit{e.g., }}
\newcommand{\CC}{C\nolinebreak\hspace{-.05em}\raisebox{.5ex}{\tiny\bf +}\nolinebreak\hspace{-.10em}\raisebox{.5ex}{\tiny\bf +}}

% units for results
\newcommand{\us}{\,$\mu$s}
\newcommand{\ms}{\,ms}
\newcommand{\KB}{\,KB}
\newcommand{\MB}{\,MB}
\newcommand{\GB}{\,GB}
\newcommand{\MHz}{\,MHz}
\newcommand{\GHz}{\,GHz}

% new latex commands:
%   Remove long section
\newcommand{\PUNT}[1]{}
\newcommand{\TABLETWO}[1]{}
%   Label work to be done
\definecolor{gray}{gray}{0.75}
\newcommand{\TODO}[1]{\textcolor{gray}{\textbf{\ [TODO:\ #1]\ }}}
\newcommand{\TR}[1]{#1}
%\newcommand{\TR}[1]{}
%\newcommand{\TODO}[1]{}
\newcommand{\FIX}[1] {\textcolor{red}{\textbf{\ [FIX:\ #1]\ }}}
%   Referencing various pieces of the document:
\newcommand{\figref}[1]{Fig.~\ref{fig:#1}}
\newcommand{\figsref}[2]{Figures~\ref{fig:#1} and~\ref{fig:#2}}
\newcommand{\figrref}[2]{Figures~\ref{fig:#1}--\ref{fig:#2}}
\newcommand{\secref}[1]{Section~\ref{sec:#1}}
\newcommand{\secsref}[2]{Sections~\ref{sec:#1} and~\ref{sec:#2}}
\newcommand{\eqnref}[1]{Eqn.~\ref{eqn:#1}}
\newcommand{\eqnsref}[2]{Equations~\ref{eqn:#1} and~\ref{eqn:#2}}
\newcommand{\eqnrref}[2]{Equations~\ref{eqn:#1}--\ref{eqn:#2}}
\newcommand{\insref}[1]{Instruction~\ref{ins:#1}}
\newcommand{\tblref}[1]{Table~\ref{tbl:#1}}
\newcommand{\appref}[1]{Appendix~\ref{app:#1}}

\newcommand{\algoref}[1]{Algorithm~\ref{algo:#1}}

% Custom hyphenation rules

%\DeclareMathOperator{\minimize}{minimize}
%\DeclareMathOperator{\st}{s.t.}
%\DeclareMathOperator*{\argmin}{arg\,min}
%\DeclareMathOperator*{\argmax}{arg\,max}
\newcommand{\argmin}{\arg\!\min}
\newcommand{\argmax}{\arg\!\max}
\newcommand{\minimize}{minimize}
\newcommand{\optimize}{optimize}
\newcommand{\ceil}[1]{\lceil #1 \rceil}
\newcommand{\floor}[1]{\lfloor #1 \rfloor}
\newcommand{\st}{s.t.}

\newcommand{\SYSTEM}{StrongBox}

% \pdfstringdefDisableCommands{
%     \def\\{}
%     \def\unskip{}
%     \def\texttt#1{<#1>}
% }

%-------------------------------------------------------------------------

\begin{document}
%% Basic setup commands
% If you don't want a title page comment out the next line and uncomment the line after it:
\maketitle
%\omittitle

% These lines can be commented out to disable the copyright/dedication/epigraph pages
\makecopyright
\makededication
To my kiddos (who at the time of writing do not yet exist): I peer over the edge
of human knowledge and reach down into that abyss of the unknown with no other
motivation than the hope that you may one day benefit. I can't wait to meet you,
haha!
% \makeepigraph

%% Make the various tables of contents
\tableofcontents
\listoffigures
\listoftables

\acknowledgments
I would first and foremost like to thank my advisor, Dr. Henry (Hank) Hoffmann.
Without your compassion, patience, conscientiousness, and cosmic wealth of
knowledge and experience, none of this work would have been possible. Thank you
for everything you've done for my students and me. You are truly a rare ally.

I'd like to thank my committee members Dr. Ariel J. Feldman (Applied Crypto),
Dr. Haryadi S. Gunawi (Filesystems), and Dr. David Cash (Theoretical Crypto) for
all their hours of hard work and tireless dedication even in the middle of a
pandemic. I learned so much studying under you. Hopefully I spelled everyone's
names correctly this time, haha!

I'd also like to acknowledge my former students Richard Anthony Alvarez and
Trevor James Medina for their huge contributions to HASCHK and to my sanity.
Equally important are Abraham Valle (a fullstack flexer!), Mark Alvarez
(remember the name, it'll be on a ballot some day), Malik Swanson, Michael
Desiderio, former students Tyrese Cook, Rakim Jazz Verner, Ephriam, Matthew,
Malcolm, Byron, Leroy, and all the others... I love you guys.

I'd be remiss if I did not acknowledge the pivotal role played by my lovely
hyper-intelligent mother (I had to get it from somewhere) Michelle Davis who,
throughout my childhood, \emph{always} had money for Computer Science workbooks
(but somehow not McDonalds!); as well as my father Bernard Dickens II for always
taking the extra effort to be there for me and always with my best interests at
heart. Thank yous also go out to: my inspiring siblings Davis and Milan; Zara
for the infinity life lessons---you were my rock, my reality hook; Jeff Holmes
of the old Chicago Broadway Armory for showing an eight-year-old me just how fun
and powerful \sout{hacking} computers can be; Von Steuben high school Computer
Science teachers Stirling Crow and Andres Hernandez for teaching me web design
and gifting me the wisdom to ``use my CS powers for good;'' my coach Audra
Anderson and the BDPA for teaching me humility and how fulfilling it is to be an
engaging teacher and effective helper; Dr. Rahmelle Thompson of the Morehouse
College Hopps Scholars program for revealing my trajectory to me; Dr. Adrienne
Raglin of the Department of Defense Army Research Laboratory for putting up with
me (twice!) and teaching me what real research looks like; neighborhood
matriarch Martha Alvarez for looking out for me during my last few years in
Chicago; the University of Chicago and all its faculty and staff for fostering a
culture of socially conscious elite intellectual pursuit; President Barack H.
Obama and Senator Elizabeth Warren for demonstrating black excellence and how to
persist nevertheless; and all my mentors, former students, and true friends over
the years for their unwavering support throughout this process. Know I would not
have made it without every last one of you.

I must give special thanks to my UChicago family: Connor. Our interactions were
usually brief, but you blazed the trail for me in particular... I was so nervous
(my default state!) but hearing you talk about your experiences
post-M.S.-defense really helped me calm down and focus. Thank you and Saeid for
all your wisdom. Anne, thanks for listening to my boring talk! Your feedback
during the group meeting really helped hone my presentation. Ivana and Will,
thank you for reaching out! I'm just a hopeless recluse. Kavon, our
conversations were fun. Huazhe would greet me on those rare occasions I entered
the office... thank you for the warm welcomes! Nikita, Yi, Santriaji, Yuli,
Tejas, Ahsan, it was my pleasure to get to know you all.

This material is based upon work supported by the National Science Foundation
under Grant No. CNS-1526304. So thank you too United States discretionary
spending budget authorized by Congress.

\abstract
The security of data at rest---widely understood as FDE or Full-Drive
Encryption---is an important concern among several in modern computer systems.
These concerns exist in contention over a set of finite resources. For instance:
a device that is battery-constrained must remain within its energy budget which
may change over time, e.g. when a device enters ``battery-saver mode'';
regardless, this device must meet certain performance guarantees or the user
experience will suffer; above all, the data on the device must be secure from
adversaries; and the device has a finite amount of drive space available. At any
given moment we trade battery life for performance, performance for security,
security for drive space, and so on. Unfortunately, designing a FDE system that
can navigate such treacherous tradeoffs efficiently, effectively, and with
respect to performance and security guarantees is entirely non-trivial. This
dissertation explores this space of tradeoffs and how we might optimize for one
concern without violating another given kernel and/or user space in-context
invariants that might shift over time.

Full-drive encryption is especially important for mobile devices because they
contain large quantities of sensitive data yet are easily lost or stolen. As
this research demonstrates, the standard approach to FDE—the AES block cipher in
XTS mode—is 3-5x slower than unencrypted storage. Authenticated encryption based
on stream ciphers is already used as a faster alternative to AES in other
contexts, such as HTTPS, but the conventional wisdom is that stream ciphers are
unsuitable for FDE. Used naively in drive encryption, stream ciphers are
vulnerable to attacks, and mitigating these attacks with on-drive metadata is
generally believed to ruin performance.

We address the difficulty of using stream ciphers for FDE with StrongBox, a
stream cipher based FDE layer that is a drop-in replacement for dm-crypt, the
standard Linux FDE module based on AES-XTS. StrongBox introduces a system design
and on-drive data structures that exploit certain properties of filesystems to
avoid costly rekeying penalties and a counter stored in trusted hardware to
protect against attacks. We implement StrongBox and SwitchCrypt on an ARM
big.LITTLE mobile processor and test its performance under multiple popular
production filesystems.

We push the envelope further with SwitchCrypt, a software mechanism that allows
us to move beyond merely making stream ciphers available for FDE. SwitchCrypt
enables practical navigation of the tradeoff space made by balancing competing
security and latency requirements via \emph{cipher switching} in space or time.
Our key insight in achieving low-overhead switching is to leverage the
overwrite-averse, append-mostly behavior of underlying solid-state storage to
trade throughput for reduced energy use and/or certain security properties.
Similar to StrongBox, we implement SwitchCrypt on an ARM big.LITTLE mobile
processor and test its performance under the popular F2FS LFS. We provide
empirical results demonstrating the conditions under which different switching
strategies are optimal through the exploration of four cases studies.

Finally, with HASCHK, we consider the same stream cipher based cryptographic
primitives in an alternative domain: data \emph{in motion} rather than at rest.
Specifically: securing data downloads over the internet. Such downloads come
with many risks, including the chance that the resource has been corrupted, or
that an attacker has replaced your desired resource with a compromised version.
The de facto standard for addressing this risk is the use of \emph{checksums}
coupled with a secure transport layer; users download a resource, compute its
checksum, and compare that with an authoritative checksum. Problems with this
approach include (1) \emph{user apathy}---for most users, calculating and
verifying the checksum is too tedious; and (2) \emph{co-hosting}---an attacker
who compromises a resource can trivially compromise a checksum hosted on the
same system. The co-hosting problem remains despite advancements in tools that
automate checksum verification and generation. In this dissertation we propose
\emph{HASCHK}, a resource verification protocol expanding on de facto
checksum-based integrity protections to defeat co-hosting while automating the
tedious parts of checksum verification to secure ``data in motion'' over the
internet.

StrongBox, SwitchCrypt, and HASCHK together demonstrate that security is indeed
a paramount concern and valid dimension with which to trade off alongside other
tier-one concerns without compromising data security or requiring obscene
performance sacrifices, all while staying within a shifting energy budget.


\mainmatter
% Main body of text follows

\chapter{Introduction} \label{chp:introduction}

In this section we make our thesis statement, describe the problem StrongBox
solves, and outline the general StrongBox approach.

\section{Thesis Statement}

With this research into filesystem, device driver, and hardware Flash
Translation Layer (FTL) based Full Disk Encryption (FDE) schemes, we primarily
consider: (1) the established wisdom in the crypto community that stream ciphers
are unsuitable for FDE and (2) exploring the tradeoff space made between total
energy use, filesystem performance, and reasonable security guarantees when
comparing specific cipher configurations. In the first case, we develop and
implement a secure approach to FDE based on stream ciphers, the proliferation of
secure hardware, and the characteristics of Log-Structured Filesystems (LFS). In
the second case, we implement a dozen stream ciphers---each exposing several
knobs---and demonstrate navigating our tradeoff space via runtime cipher
switching while maintaining reasonable security guarantees. We then present a
formal analysis of our system's security guarantees.

\section{Problem Description}

Full-drive encryption (FDE)\footnote{The common term is full-\emph{disk}
encryption, but this work targets SSDs, so we use \emph{drive}.} is an essential
technique for protecting the privacy of data at rest. For mobile devices,
maintaining data privacy is especially important as these devices contain
sensitive personal and financial data yet are easily lost or stolen. The current
standard for securing data at rest is to use the AES cipher in XTS
mode~\cite{XTS, NISTXTS}. Unfortunately, employing AES-XTS increases read/write
latency by 3--5$\times$ compared to unencrypted storage.

It is well known that authenticated encryption using \emph{stream}
ciphers---such as ChaCha20~\cite{ChaCha20}---is faster than using AES. Indeed,
Google made the case for stream ciphers over AES, switching HTTPS connections on
Chrome for Android to use a stream cipher for better
performance~\cite{google-blog}. Stream ciphers are not used for FDE, however,
for two reasons: (1) confidentiality and (2) performance. First, when applied
naively to stored data, stream ciphers are trivially vulnerable to
attacks---including \emph{many-time pad and rollback attacks}---that reveal the
plaintext by overwriting a secure storage location with the same key. Second, it
has been assumed that adding the meta-data required to resist these attacks
would ruin the stream cipher's performance advantage. Thus, the conventional
wisdom is that FDE necessarily incurs the overhead of AES-XTS or a similar
primitive.

We argue that two technological shifts in mobile device hardware overturn this
conventional wisdom, enabling confidential, high-performance storage with stream
ciphers. First, these devices commonly employ solid-state storage with Flash
Translation Layers (FTL), which operate similarly to Log-structured File Systems
(LFS)~\cite{LFS,F2FS,NILFS}. Second, mobile devices now support trusted
hardware, such as Trusted Execution Environments (TEE)~\cite{TEE,TrustZone} and
secure storage areas~\cite{eMMC-standard}. FTLs and LFSes are used to limit
sector/cell overwrites, hence extending the life of the drive. Most writes
simply appended to a log, reducing the occurrence of overwrites and the chance
for attacks. The presence of secure hardware means that drive encryption modules
have access to persistent, monotonically increasing counters that can be used to
prevent rollback attacks when overwrites do occur.

Given these trends, we propose StrongBox, a new method for securing data at
rest. StrongBox is a drop-in replacement for AES-XTS-backed FDE such as
dm-crypt~\cite{dmcrypt}; i.e. it requires no interface changes. The primary
challenge is that even with a FTL or LFS running above an SSD, filesystem blocks
will occasionally be overwritten; e.g. by segment cleaning or \emph{garbage
collection}. StrongBox overcomes this challenge by using a fast stream cipher
for confidentiality and performance with integrity preserving Message
Authentication Codes~\cite{MAC} or ``MAC tags'' and a secure, persistent
hardware counter to ensure integrity and prevent attacks. \emph{StrongBox's main
contribution is a system design enabling the first confidential,
high-performance drive encryption based on a stream cipher.}

\section{Related Work} \label{sec:hc-related}

In this section, we examine prior approaches to guaranteeing resource integrity
over the internet. We then highlight some drawbacks to these approaches and how
HASCHK differs. \\

\noindent\textbf{Anti-malware software, heuristics, and blacklists.}
Anti-malware software is a heuristic-based program designed for the specific
purpose of detecting and removing various kinds of malware. However, updates to
anti-malware definitions often lag behind or occur in response to the release of
crippling malware. For example, during the 2017 compromise of the HandBrake
distribution mirror, users who first ran the compromised HandBrake image through
\textit{VirusTotal}---a web service that will run a resource through several
dozen popular anti-malware products---received a report claiming no infections
were detected despite the presence of the Proton malware~\cite{SCA-HB1}. In the
2012 compromise of SourceForge's CDN, the malicious changes to the phpmyAdmin
image do not appear as malware to anti-malware software~\cite{SCA-PMA1}.

Similarly, all modern browsers employ heuristic and blacklist-based detection
and prevention schemes in an attempt to protect users from malicious content on
the internet. The warnings generated by browser-based heuristics and blacklists
are also reactive rather than proactive; hence, they are generally ineffective
at detecting active or novel attacks on the integrity of the resources
downloaded over the internet.

On the other hand, HASCHK relies on no heuristics or blacklists and is not
anti-malware software. HASCHK is a protocol for automating checksum
verification of resources. This ensures download integrity---that a user is
receiving the expected resource a provider is advertising---not that the
expected resource is not malware. \\

\noindent\textbf{Link Fingerprints and Subresource Integrity.} The Link
Fingerprints (LF) draft describes an early HTML hyperlinks and URI based
resource integrity verification scheme that ``provides a backward-compatible
technique for resource providers to ensure that the resource originally
referenced is the same as the resource retrieved by an end user.''~\cite{LF}.
The World Wide Web Consortium's (W3C) Subresource Integrity (SRI) describes a
similar HTML-based scheme designed exclusively with CDNs and web assets in mind.

Like HASCHK, both LF and SRI employ cryptographic digests to ensure no
changes of any kind have been made to a resource~\cite{SRI}. Unlike HASCHK,
LF and SRI apply \emph{only to resources referenced by script and link HTML
elements}; HASCHK, on the other hand, can ensure the integrity of \emph{any
arbitrary resource downloaded over the internet}, even outside of HTML web pages
and browser software. Further, the checksums contained in the HTML source must
be accurate for SRI to work. If the system \emph{behind} the CDN is compromised,
the attacker can alter the HTML and inject a malicious checksum or even strip
checksums from the HTML entirely. With HASCHK, however, an attacker would
additionally have to compromise the separate backend system that advertises the
provider's resources, thus raising the bar. \\

\noindent\textbf{Content-MD5 Header.} The Content-MD5 header field is a
deprecated email and HTTP header that delivers a checksum similar to those used
by Subresource Integrity. It was removed from the HTTP/1.1 specification because
of the inconsistent implementation of partial response handling between
vendors~\cite{HTTP1.1}. Further, the header could be easily stripped off or
modified by proxies and other intermediaries~\cite{MD5Header}. HASCHK
exhibits none of these weaknesses. \\

\noindent\textbf{Deterministic Build Systems and Binary Transparency.} A
deterministic build system is one that, when given the same source, will
deterministically output the same binary on every run. For example, many
packages in Debian~\cite{ReproBuildsDebian} and Arch Linux can be rebuilt from
source to yield an identical byte for byte result~\cite{ReproBuilds}, allowing
for verification of the \emph{Integration} and perhaps \emph{Development} supply
chain phases (see \tblref{attacks}). Further, using a merkle
tree~\cite{MerkleTree} or similar construction, an additional chain of trust can
be established that allows for public verification of the \emph{Deployment},
\emph{Maintenance}, and \emph{Retirement} supply chain phases. Companies such as
Mozilla refer to the latter as ``Binary Transparency.''

Like HASCHK, binary transparency establishes a public verification scheme
that allows third party consumers access to a listing of source updates
advertised by a provider~\cite{BinaryTransparency}. Consumers can leverage
deterministic build systems and Binary Transparency together to ensure their
software is the same software deployed to every other system. Unlike HASCHK,
binary transparency only allows a user to verify the integrity of \emph{source
updates to binaries}; our protocol allows a user to verify the integrity of
\emph{any arbitrary resource} while specifically addressing co-hosting. \\

\noindent\textbf{Stickler and Cherubini et al.} Stickler~\cite{Stickler} by Levy
et al. is an automated JavaScript-based stand-in for SRI for protecting the
integrity of web application files hosted on CDNs. Stickler does not require any
modifications to the client (\ie{ a frontend}), instead delivering a bootloader
to load and verify resources signed before the fact. However, as it was designed
to stand-in for SRI, Stickler inherits some of SRI's limitations. Specifically:
Stickler was not designed to protect arbitrary resource downloads and, if the
publisher's server is compromised, Stickler's bootloader can be stripped out of
the initial HTTP response altogether. Something like this is not possible with
HASCHK.

The automated checksum verification approach described by Cherubini et
al.~\cite{Cherubini}, also based on SRI, is similarly vulnerable to (and relies
upon) co-hosting. Cherubini's browser extension works by both looking for
embedded checksums in download links (SRI) and extracting hexadecimal strings
that look like checksums directly from the HTML source. An attacker, after
compromising the resource file, need only modify the provider's HTML file to
inject a corrupted ``integrity'' attribute containing a checksum matching that
corrupted resource, causing Cherubini's extension to misreport the dangerous
download as safe~\cite{Cherubini}. Additionally, Cherubini's extension: (1) does
not alert users when corresponding authoritative checksums are not found, which
means an attacker can simply strip all checksums from the server response to
pass off compromised resources to users; (2) considers a download ``safe'' so
long as \emph{any checksum found on the page matches it}, which means an
attacker can just inject the compromised checksum somewhere in the HTML source
alongside the legitimate checksum to similarly pass off compromised resources to
users; (3) does not support direct downloads, \ie{ when a user enters a
resource's URI into the browser manually rather than click a hyperlink}. None of
this is a problem for HASCHK.


\chapter{StrongBox: Using Stream Ciphers for High Performance Full Drive Encryption} \label{chp:strongbox}
\begin{abstract}

Full-drive encryption (FDE) is especially important for mobile devices because
they contain large quantities of sensitive data yet are easily lost or stolen.
Unfortunately, the standard approach to FDE---the AES block cipher in XTS
mode---is 3--5x slower than unencrypted storage. Authenticated encryption based on
stream ciphers is already used as a faster alternative to AES in other contexts,
such as HTTPS, but the conventional wisdom is that stream ciphers are unsuitable
for FDE. Used naively in drive encryption, stream ciphers are vulnerable to
attacks, and mitigating these attacks with on-drive metadata is generally
believed to ruin performance.

In this paper, we argue that recent developments in mobile hardware invalidate
this assumption, making it possible to use fast stream ciphers for FDE. Modern
mobile devices employ solid-state storage with Flash Translation Layers (FTL),
which operate similarly to Log-structured File Systems (LFS). They also include
trusted hardware such as Trusted Execution Environments (TEEs) and secure
storage areas. Leveraging these two trends, we propose StrongBox, a stream
cipher-based FDE layer that is a drop-in replacement for dm-crypt, the standard
Linux FDE module based on AES-XTS. StrongBox introduces a system design and
on-drive data structures that exploit LFS's lack of overwrites to avoid costly
rekeying and a counter stored in trusted hardware to protect against attacks. We
implement StrongBox on an ARM big.LITTLE mobile processor and test its
performance under multiple popular production LFSes. We find that StrongBox
improves read performance by as much as $2.36\times$ ($1.72\times$ on average)
while offering stronger integrity guarantees.

\end{abstract}


\chapter{SwitchCrypt: Navigating Tradeoffs in Stream Cipher Based Full Drive Encryption} \label{chp:switchcrypt}
\begin{abstract}

Recent work on Full Drive Encryption shows that stream ciphers achieve
significantly improved performance over block ciphers while offering stronger
security guarantees. However, optimizing for performance often conflicts with
other key concerns like energy usage and desired security properties. In this
paper we present SwitchCrypt, a software mechanism that navigates the tradeoff
space made by balancing competing security and latency requirements via
\emph{cipher switching} in space or time. Our key insight in achieving
low-overhead switching is to leverage the overwrite-averse, append-mostly
behavior of underlying solid-state storage to trade throughput for reduced
energy use and/or certain security properties. We implement SwitchCrypt on an
ARM big.LITTLE mobile processor and test its performance under the popular F2FS
LFS. We provide empirical results demonstrating the conditions under which
different switching strategies are optimal through the exploration of four cases
studies. In one study, where we require the filesystem to react to a shrinking
energy budget by switching ciphers, we find that SwitchCrypt achieves up to a
3.3x total energy use reduction compared to a static approach using only the
Freestyle stream cipher. In another case, where we allow the user to manually
switch between ChaCha20 and Freestyle stream ciphers dynamically, we achieve a
1.6x to 4.8x reduction in I/O latency compared to prior static approaches.

\end{abstract}


\chapter{HASCHK} \label{chp:haschk}
\input{5_haschk}

\chapter{Conclusion} \label{chp:conclusion}

The conventional wisdom is that securing data at rest requires one must pay the
high performance overhead of encryption with AES is XTS mode. This work shows
that technological trends overturn this conventional wisdom: log-structured file
systems and hardware support for secure counters make it practical to use a
stream cipher to secure data at rest. We demonstrate this practicality through
our implementation of StrongBox which uses the ChaCha20 stream cipher and the
Poly1305 MAC to provide secure storage and can be used as a drop-in replacement
for dm-crypt.

Our empirical results show that under F2FS---a modern, industrial-strength
Log-structured file system---StrongBox provides upwards of $2\times$ improvement
on read performance and average $1.27\times$ improvement on write performance
compared to dm-crypt. Further, our results show that F2FS plus StrongBox
provides a higher performance replacement for Ext4 backed with dm-crypt. We make
our implementation of StrongBox available open source so that others can extend
it or compare to it.\footnote{\StrongBoxURI} Our hope is that this work
motivates further exploration of fast stream ciphers as replacements for AES-XTS
for securing data at rest.

Further, this work advocates for a more flexible approach to FDE where the
storage system can dynamically adjust the tradeoffs between security and
latency/energy. To support this vision of agile encryption, we proposed an
interface that allows multiple stream ciphers with different input and output
characteristics to be composed in a generic manner. We have identified three
strategies for using this interface to switch ciphers dynamically and with low
overhead. We have also proposed a quantification framework for determining when
to use one cipher over another. Our case studies show how different strategies
can be used to optimize for different goals in practice. We believe that agile
encryption will become increasingly important as successful systems are
increasingly required to balance conflicting operational requirements. We hope
that this work inspires further research in achieving this balance. Our work is
publicly available open-source\footnoteref{note1}.

When it comes to data in motion---specifically, downloading arbitrary resources
over the internet---it is indeed a risky endeavor. Resource integrity and other
Supply Chain Attacks are becoming more frequent and their impact more widely
felt. In this work, we showed that the de facto standard for protecting the
integrity of arbitrary resources on the internet---the use of
\emph{checksums}---is insufficient and often ineffective. We presented HASCHK, a
practical resource verification protocol that automates the tedious parts of
checksum verification while leveraging pre-existing high availability systems to
ensure resources and their checksums are not vulnerable to co-hosting. Further,
we demonstrated the effectiveness and practicality of our approach versus
real-world resource integrity attacks in a production application.

The results of our final evaluation show that our approach is more effective
than checksums and prior work mitigating integrity attacks for arbitrary
resources on the internet. Further, we show HASCHK is capable of guarding
against a variety of attacks, is deployable at scale for providers that already
maintain a DNS presence, and can be deployed without fear of adversely affecting
the user experience of clients that are not HASCHK-aware. Though not a panacea,
we believe our protocol significantly raises the bar for the attacker. We intend
to continue developing our extension and we make it available to a wide audience
(see \cref{app:availability}).

\section{Future Work}

As it stands now, our StrongBox and SwitchCrypt implementations are single
threaded. The cryptographic systems we compared to StrongBox and SwitchCrypt and
all multi-threaded, perhaps making our performance wins more significant. What
performance gains (i.e. more slack) might be available if we parallelize
portions of StrongBox and SwitchCrypt source? Examples include: 1) given that
nuggets are all independent agents, new switching strategies might be developed
that leverage any speedups in parallelizing the encryption and decryption
processes and 2) garbage collection and cipher switching can happen in
background worker threads and could even leverage a scheduler and 3) StrongBox
and SwitchCrypt could be integrated into F2FS directly, sharing its
threading/scheduling model in a production implementation; currently, StrongBox
and SwitchCrypt are implemented on top of the BUSE emulation device.

\renewcommand\thechapter{A}
\chapter{Code Availability} \label{app:availability}

This appendix provides links for the StrongBox I and StrongBox II source code.
The two libraries/APIs that our StrongBox implementations rely on are
additionally provided. Note that resource locations, URLs, frameworks,
interfaces, etc. will likely change overtime, while this text remains static.

You can find instructions on how to build, test, and modify the StrongBox source
in the provided README documentation linked below.

\begin{table}
    \centering
    \caption{Provides URLs for the products yielded and/or used by this research.}
    \begin{tabular}{|r|c|l|l}
        \cline{1-3}
        \textbf{Project Source} & \textbf{Language} & \textbf{URL} & \\
        \cline{1-3}
        StrongBox I Source & C & https://github.com/research/buselfs-public & \\
        StrongBox II Source & C & https://github.com/research/buselfs2-public & \\
        SBD Merkle Tree Implementation & C & https://github.com/IAIK/secure-block-device & \\
        Block Device in User Space (BUSE) & C/Shell & https://github.com/acozzette/BUSE & \\
        \cline{1-3}
    \end{tabular}
\end{table}



% Format a LaTeX bibliography
\clearpage
\printbibliography

% Figures and tables, if you decide to leave them to the end
%\input{figure}
%\input{table}

\end{document}


%% Use these commands to set a dedication and epigraph text
%\dedication{Dedication Text}
%\epigraph{Epigraph Text}

% Referencing various pieces of the document:
\newcommand{\figref}[1]{Fig.~\ref{fig:#1}}
\newcommand{\figsref}[2]{Figures~\ref{fig:#1} and~\ref{fig:#2}}
\newcommand{\figrref}[2]{Figures~\ref{fig:#1}--\ref{fig:#2}}
\newcommand{\secref}[1]{Section~\ref{sec:#1}}
\newcommand{\secsref}[2]{Sections~\ref{sec:#1} and~\ref{sec:#2}}
\newcommand{\eqnref}[1]{Eqn.~\ref{eqn:#1}}
\newcommand{\eqnsref}[2]{Equations~\ref{eqn:#1} and~\ref{eqn:#2}}
\newcommand{\eqnrref}[2]{Equations~\ref{eqn:#1}--\ref{eqn:#2}}
\newcommand{\insref}[1]{Instruction~\ref{ins:#1}}
\newcommand{\tblref}[1]{Table~\ref{tbl:#1}}
\newcommand{\chapref}[1]{Chapter~\ref{chp:#1}}
\newcommand{\appref}[1]{Appendix~\ref{app:#1}}

\begin{document}
%% Basic setup commands
% If you don't want a title page comment out the next line and uncomment the line after it:
\maketitle
%\omittitle

% These lines can be commented out to disable the copyright/dedication/epigraph pages
%\makecopyright
%\makededication
%\makeepigraph

%% Make the various tables of contents
\tableofcontents
%\listoffigures
\listoftables

%\acknowledgments
% Enter Acknowledgements here

\abstract
\abstract
The security of data at rest---widely understood as FDE or Full-Drive
Encryption---is an important concern among several in modern computer systems.
These concerns exist in contention over a set of finite resources. For instance:
a device that is battery-constrained must remain within its energy budget which
may change over time, e.g. when a device enters ``battery-saver mode'';
regardless, this device must meet certain performance guarantees or the user
experience will suffer; above all, the data on the device must be secure from
adversaries; and the device has a finite amount of drive space available. At any
given moment we trade battery life for performance, performance for security,
security for drive space, and so on. Unfortunately, designing a FDE system that
can navigate such treacherous tradeoffs efficiently, effectively, and with
respect to performance and security guarantees is entirely non-trivial. This
dissertation explores this space of tradeoffs and how we might optimize for one
concern without violating another given kernel and/or user space in-context
invariants that might shift over time.

Full-drive encryption is especially important for mobile devices because they
contain large quantities of sensitive data yet are easily lost or stolen. As
this research demonstrates, the standard approach to FDE—the AES block cipher in
XTS mode—is 3-5x slower than unencrypted storage. Authenticated encryption based
on stream ciphers is already used as a faster alternative to AES in other
contexts, such as HTTPS, but the conventional wisdom is that stream ciphers are
unsuitable for FDE. Used naively in drive encryption, stream ciphers are
vulnerable to attacks, and mitigating these attacks with on-drive metadata is
generally believed to ruin performance.

We address the difficulty of using stream ciphers for FDE with StrongBox, a
stream cipher based FDE layer that is a drop-in replacement for dm-crypt, the
standard Linux FDE module based on AES-XTS. StrongBox introduces a system design
and on-drive data structures that exploit certain properties of filesystems to
avoid costly rekeying penalties and a counter stored in trusted hardware to
protect against attacks. We implement StrongBox and SwitchCrypt on an ARM
big.LITTLE mobile processor and test its performance under multiple popular
production filesystems.

We push the envelope further with SwitchCrypt, a software mechanism that allows
us to move beyond merely making stream ciphers available for FDE. SwitchCrypt
enables practical navigation of the tradeoff space made by balancing competing
security and latency requirements via \emph{cipher switching} in space or time.
Our key insight in achieving low-overhead switching is to leverage the
overwrite-averse, append-mostly behavior of underlying solid-state storage to
trade throughput for reduced energy use and/or certain security properties.
Similar to StrongBox, we implement SwitchCrypt on an ARM big.LITTLE mobile
processor and test its performance under the popular F2FS LFS. We provide
empirical results demonstrating the conditions under which different switching
strategies are optimal through the exploration of four cases studies.

Finally, with HASCHK, we consider the same stream cipher based cryptographic
primitives in an alternative domain: data \emph{in motion} rather than at rest.
Specifically: securing data downloads over the internet. Such downloads come
with many risks, including the chance that the resource has been corrupted, or
that an attacker has replaced your desired resource with a compromised version.
The de facto standard for addressing this risk is the use of \emph{checksums}
coupled with a secure transport layer; users download a resource, compute its
checksum, and compare that with an authoritative checksum. Problems with this
approach include (1) \emph{user apathy}---for most users, calculating and
verifying the checksum is too tedious; and (2) \emph{co-hosting}---an attacker
who compromises a resource can trivially compromise a checksum hosted on the
same system. The co-hosting problem remains despite advancements in tools that
automate checksum verification and generation. In this dissertation we propose
\emph{HASCHK}, a resource verification protocol expanding on de facto
checksum-based integrity protections to defeat co-hosting while automating the
tedious parts of checksum verification to secure ``data in motion'' over the
internet.

StrongBox, SwitchCrypt, and HASCHK together demonstrate that security is indeed
a paramount concern and valid dimension with which to trade off alongside other
tier-one concerns without compromising data security or requiring obscene
performance sacrifices, all while staying within a shifting energy budget.


\mainmatter
% Main body of text follows

\chapter{Introduction} \label{chp:introduction}

In this section we make our thesis statement, describe the problem StrongBox
solves, and outline the general StrongBox approach.

\section{Thesis Statement}

With this research into filesystem, device driver, and hardware Flash
Translation Layer (FTL) based Full Disk Encryption (FDE) schemes, we primarily
consider: (1) the established wisdom in the crypto community that stream ciphers
are unsuitable for FDE and (2) exploring the tradeoff space made between total
energy use, filesystem performance, and reasonable security guarantees when
comparing specific cipher configurations. In the first case, we develop and
implement a secure approach to FDE based on stream ciphers, the proliferation of
secure hardware, and the characteristics of Log-Structured Filesystems (LFS). In
the second case, we implement a dozen stream ciphers---each exposing several
knobs---and demonstrate navigating our tradeoff space via runtime cipher
switching while maintaining reasonable security guarantees. We then present a
formal analysis of our system's security guarantees.

\section{Problem Description}

Full-drive encryption (FDE)\footnote{The common term is full-\emph{disk}
encryption, but this work targets SSDs, so we use \emph{drive}.} is an essential
technique for protecting the privacy of data at rest. For mobile devices,
maintaining data privacy is especially important as these devices contain
sensitive personal and financial data yet are easily lost or stolen. The current
standard for securing data at rest is to use the AES cipher in XTS
mode~\cite{XTS, NISTXTS}. Unfortunately, employing AES-XTS increases read/write
latency by 3--5$\times$ compared to unencrypted storage.

It is well known that authenticated encryption using \emph{stream}
ciphers---such as ChaCha20~\cite{ChaCha20}---is faster than using AES. Indeed,
Google made the case for stream ciphers over AES, switching HTTPS connections on
Chrome for Android to use a stream cipher for better
performance~\cite{google-blog}. Stream ciphers are not used for FDE, however,
for two reasons: (1) confidentiality and (2) performance. First, when applied
naively to stored data, stream ciphers are trivially vulnerable to
attacks---including \emph{many-time pad and rollback attacks}---that reveal the
plaintext by overwriting a secure storage location with the same key. Second, it
has been assumed that adding the meta-data required to resist these attacks
would ruin the stream cipher's performance advantage. Thus, the conventional
wisdom is that FDE necessarily incurs the overhead of AES-XTS or a similar
primitive.

We argue that two technological shifts in mobile device hardware overturn this
conventional wisdom, enabling confidential, high-performance storage with stream
ciphers. First, these devices commonly employ solid-state storage with Flash
Translation Layers (FTL), which operate similarly to Log-structured File Systems
(LFS)~\cite{LFS,F2FS,NILFS}. Second, mobile devices now support trusted
hardware, such as Trusted Execution Environments (TEE)~\cite{TEE,TrustZone} and
secure storage areas~\cite{eMMC-standard}. FTLs and LFSes are used to limit
sector/cell overwrites, hence extending the life of the drive. Most writes
simply appended to a log, reducing the occurrence of overwrites and the chance
for attacks. The presence of secure hardware means that drive encryption modules
have access to persistent, monotonically increasing counters that can be used to
prevent rollback attacks when overwrites do occur.

Given these trends, we propose StrongBox, a new method for securing data at
rest. StrongBox is a drop-in replacement for AES-XTS-backed FDE such as
dm-crypt~\cite{dmcrypt}; i.e. it requires no interface changes. The primary
challenge is that even with a FTL or LFS running above an SSD, filesystem blocks
will occasionally be overwritten; e.g. by segment cleaning or \emph{garbage
collection}. StrongBox overcomes this challenge by using a fast stream cipher
for confidentiality and performance with integrity preserving Message
Authentication Codes~\cite{MAC} or ``MAC tags'' and a secure, persistent
hardware counter to ensure integrity and prevent attacks. \emph{StrongBox's main
contribution is a system design enabling the first confidential,
high-performance drive encryption based on a stream cipher.}

\section{Related Work} \label{sec:hc-related}

In this section, we examine prior approaches to guaranteeing resource integrity
over the internet. We then highlight some drawbacks to these approaches and how
HASCHK differs. \\

\noindent\textbf{Anti-malware software, heuristics, and blacklists.}
Anti-malware software is a heuristic-based program designed for the specific
purpose of detecting and removing various kinds of malware. However, updates to
anti-malware definitions often lag behind or occur in response to the release of
crippling malware. For example, during the 2017 compromise of the HandBrake
distribution mirror, users who first ran the compromised HandBrake image through
\textit{VirusTotal}---a web service that will run a resource through several
dozen popular anti-malware products---received a report claiming no infections
were detected despite the presence of the Proton malware~\cite{SCA-HB1}. In the
2012 compromise of SourceForge's CDN, the malicious changes to the phpmyAdmin
image do not appear as malware to anti-malware software~\cite{SCA-PMA1}.

Similarly, all modern browsers employ heuristic and blacklist-based detection
and prevention schemes in an attempt to protect users from malicious content on
the internet. The warnings generated by browser-based heuristics and blacklists
are also reactive rather than proactive; hence, they are generally ineffective
at detecting active or novel attacks on the integrity of the resources
downloaded over the internet.

On the other hand, HASCHK relies on no heuristics or blacklists and is not
anti-malware software. HASCHK is a protocol for automating checksum
verification of resources. This ensures download integrity---that a user is
receiving the expected resource a provider is advertising---not that the
expected resource is not malware. \\

\noindent\textbf{Link Fingerprints and Subresource Integrity.} The Link
Fingerprints (LF) draft describes an early HTML hyperlinks and URI based
resource integrity verification scheme that ``provides a backward-compatible
technique for resource providers to ensure that the resource originally
referenced is the same as the resource retrieved by an end user.''~\cite{LF}.
The World Wide Web Consortium's (W3C) Subresource Integrity (SRI) describes a
similar HTML-based scheme designed exclusively with CDNs and web assets in mind.

Like HASCHK, both LF and SRI employ cryptographic digests to ensure no
changes of any kind have been made to a resource~\cite{SRI}. Unlike HASCHK,
LF and SRI apply \emph{only to resources referenced by script and link HTML
elements}; HASCHK, on the other hand, can ensure the integrity of \emph{any
arbitrary resource downloaded over the internet}, even outside of HTML web pages
and browser software. Further, the checksums contained in the HTML source must
be accurate for SRI to work. If the system \emph{behind} the CDN is compromised,
the attacker can alter the HTML and inject a malicious checksum or even strip
checksums from the HTML entirely. With HASCHK, however, an attacker would
additionally have to compromise the separate backend system that advertises the
provider's resources, thus raising the bar. \\

\noindent\textbf{Content-MD5 Header.} The Content-MD5 header field is a
deprecated email and HTTP header that delivers a checksum similar to those used
by Subresource Integrity. It was removed from the HTTP/1.1 specification because
of the inconsistent implementation of partial response handling between
vendors~\cite{HTTP1.1}. Further, the header could be easily stripped off or
modified by proxies and other intermediaries~\cite{MD5Header}. HASCHK
exhibits none of these weaknesses. \\

\noindent\textbf{Deterministic Build Systems and Binary Transparency.} A
deterministic build system is one that, when given the same source, will
deterministically output the same binary on every run. For example, many
packages in Debian~\cite{ReproBuildsDebian} and Arch Linux can be rebuilt from
source to yield an identical byte for byte result~\cite{ReproBuilds}, allowing
for verification of the \emph{Integration} and perhaps \emph{Development} supply
chain phases (see \tblref{attacks}). Further, using a merkle
tree~\cite{MerkleTree} or similar construction, an additional chain of trust can
be established that allows for public verification of the \emph{Deployment},
\emph{Maintenance}, and \emph{Retirement} supply chain phases. Companies such as
Mozilla refer to the latter as ``Binary Transparency.''

Like HASCHK, binary transparency establishes a public verification scheme
that allows third party consumers access to a listing of source updates
advertised by a provider~\cite{BinaryTransparency}. Consumers can leverage
deterministic build systems and Binary Transparency together to ensure their
software is the same software deployed to every other system. Unlike HASCHK,
binary transparency only allows a user to verify the integrity of \emph{source
updates to binaries}; our protocol allows a user to verify the integrity of
\emph{any arbitrary resource} while specifically addressing co-hosting. \\

\noindent\textbf{Stickler and Cherubini et al.} Stickler~\cite{Stickler} by Levy
et al. is an automated JavaScript-based stand-in for SRI for protecting the
integrity of web application files hosted on CDNs. Stickler does not require any
modifications to the client (\ie{ a frontend}), instead delivering a bootloader
to load and verify resources signed before the fact. However, as it was designed
to stand-in for SRI, Stickler inherits some of SRI's limitations. Specifically:
Stickler was not designed to protect arbitrary resource downloads and, if the
publisher's server is compromised, Stickler's bootloader can be stripped out of
the initial HTTP response altogether. Something like this is not possible with
HASCHK.

The automated checksum verification approach described by Cherubini et
al.~\cite{Cherubini}, also based on SRI, is similarly vulnerable to (and relies
upon) co-hosting. Cherubini's browser extension works by both looking for
embedded checksums in download links (SRI) and extracting hexadecimal strings
that look like checksums directly from the HTML source. An attacker, after
compromising the resource file, need only modify the provider's HTML file to
inject a corrupted ``integrity'' attribute containing a checksum matching that
corrupted resource, causing Cherubini's extension to misreport the dangerous
download as safe~\cite{Cherubini}. Additionally, Cherubini's extension: (1) does
not alert users when corresponding authoritative checksums are not found, which
means an attacker can simply strip all checksums from the server response to
pass off compromised resources to users; (2) considers a download ``safe'' so
long as \emph{any checksum found on the page matches it}, which means an
attacker can just inject the compromised checksum somewhere in the HTML source
alongside the legitimate checksum to similarly pass off compromised resources to
users; (3) does not support direct downloads, \ie{ when a user enters a
resource's URI into the browser manually rather than click a hyperlink}. None of
this is a problem for HASCHK.

\chapter{StrongBox I, II, III} \label{chp:strongbox}

In this section we describe the three primary components of the StrongBox
research project, and define a timeline of completion.

\section{Project Timeline}

These are dates we expect certain research milestones to be accomplished:

\begin{enumerate}
    \item \textbf{4/2019} Complete use case implementation and case studies.
    \item \textbf{5/2019} Literature review on FDE theory and formal
    organization.
    \item \textbf{5-6/2019} StrongBox II cipher switching draft paper is
    completed.
    \item \textbf{6/2019} Formal theoretical framework to model StrongBox
    security guarantees.
    \item \textbf{12/2019} Analysis of security model and guarantees versus
    AES-XTS and other schemes.
    \item \textbf{2/2020} StrongBox III (theoretical security framework) draft
    paper is completed.
\end{enumerate}

\section{StrongBox I: Confidentiality, Integrity, and Performance using Stream Ciphers}

StrongBox I is the original published work demonstrating how recent developments
in hardware coupled with core insights about filesystem behavior invalidate the
assumption that practically mitigating attacks against high performance stream
ciphers for FDE is infeasible.

With this research, it was shown that recent developments in mobile hardware
invalidate the assumption that stream ciphers are unsuitable for FDE, making it
possible to take advantage of fast stream ciphers. Modern mobile devices employ
solid-state storage with Flash Translation Layers (FTL), which operate similarly
to Log-structured File Systems (LFS). They also include trusted hardware such as
Trusted Execution Environments (TEEs) and secure storage areas. StrongBox I,
implemented with the ChaCha20 stream cipher, leveraged these two trends to
outperform dm-crypt, the de-facto Linux FDE endpoint.

\subsection{Completed Work}

\begin{itemize}
    \item \textbf{Published.} \textit{StrongBox: Confidentiality, Integrity, and
    Performance using Stream Ciphers for Full-Drive Encryption} was published in
    the Proceedings of the Twenty-Third International Conference on
    Architectural Support for Programming Languages and Operating Systems
    (ASPLOS'18).

    \item \textbf{Implemented.} A prototype of the StrongBox approach was
    implemented in C. The full 5000+ LoC source for StrongBox I is available
    open source (see \appref{availability}).

    \item \textbf{Patented.} The StrongBox approach has been patented by the
    University of Chicago, pending prosecution and non-provisional filing.

    \item \textbf{Inspired.} Using stream ciphers for Full-Drive Encryption
    motivated research in a similar vein: using stream ciphers with highly
    available systems to secure arbitrary resources on the internet (HASCHK).
\end{itemize}

\section{StrongBox II: Energy, Security, and Performance Tradeoffs with Cipher Switching}

With StrongBox II, we evaluate a new StrongBox implementation capable of using
ciphers beyond ChaCha20 and AES-CTR; specifically, the following profile 1
stream ciphers were added to StrongBox: Sosemanuk, HC-128, Rabbit, Salsa20,
Salsa12, and Salsa8, ChaCha12, ChaCha8, Freestyle, and ARM Neon SIMD accelerated
versions of ChaCha.

The eSTREAM portfolio ciphers fall into two profiles. Profile 1 contains stream
ciphers more suitable for software applications with high throughput
requirements. Profile 2 stream ciphers are particularly suitable for hardware
applications with restricted resources such as limited storage, gate count, or
power consumption. This research does not consider profile 2 stream ciphers
offered by the eSTREAM portfolio.

\subsection{Completed Work}

\begin{itemize}
    \item \textbf{Cipher abstraction and eSTREAM implementations.} With
    StrongBox II, we abstracted the cipher interface into an independent
    subsystem such that StrongBox II functions with any stream or block cipher,
    including the original ChaCha20. We also included an implementation of the
    Freestyle randomized output stream cipher for consideration with FDE,
    similarly implemented in C. The full source code for StrongBox II is
    available alongside the original (see \appref{availability}).

    \item \textbf{Cipher switching framework and tradeoff exploration.} We
    performed experiments to quantify the performance and security properties of
    the various ciphers we implemented. From that, we engineered a ``cipher
    switching'' abstraction allowing StrongBox to switch (offline) between any
    cipher in its library. Further, by studying the contention between
    StrongBox's observed \emph{energy} use, the \emph{security} of the StrongBox
    construction given a certain cipher configuration, and the
    \emph{performance} of the construction under a given workload, we modeled a
    navigable tradeoff space where manipulating magnitude in one dimension
    yields a non-linear change in another.

    \item \textbf{Cipher switching strategies.} We implemented five ``switching
    strategies'' based on our experimental observations. These strategies allow
    StrongBox to navigate the aforementioned tradeoff space both offline
    \emph{and online}.
\end{itemize}

\subsection{Current Work}

\begin{itemize}
    \item \textbf{4-5/2019}
        \begin{itemize}
            \item Complete experimental evaluation of (online) cipher switching
            strategies.
            \item Finish use case framework implementation and determine worthy
            use case implementations based on the aforementioned evaluation.
            \item Complete study and experimental evaluation of worthy use
            cases.
        \end{itemize}
    \item \textbf{5-6/2019}
        \begin{itemize}
            \item Collate research into paper: \textit{StrongBox II: A Study of
            Practical Tradeoffs Between Energy, Performance, and Security}.
        \end{itemize}
\end{itemize}

\subsection{Considerations}

At this point in the project, we have established a useful set of pareto curves,
the cipher switching strategy implementations are practical, and our use cases
are built on top of those switching strategies. Our biggest concern is accurate
and meaningful energy readings. Integrating the energy use telemetry (i.e.
Energymon) directly into the StrongBox core instead of at the test fixture level
along with other tweaks and checks should address this. However, even without a
robust set of metrics on energy use, the performance-security curve alone
motivates the switching strategies and, in turn, the use cases.

\section{StrongBox III: Theoretical Framework for Analyzing StrongBox FDE}

With the final component of this research, at minimum we want: (1) define a set of
formal FDE security notions in context, (2) define the cryptographic goals and
guarantees of StrongBox, (3) introduce a theoretical framework and models to
evaluate StrongBox, (4) potentially explore AES-XTS security guarantees versus
StrongBox guarantees.

\subsection{Current Work}

\begin{itemize}
    \item \textbf{5-6/2019}
        \begin{itemize}
            \item Complete review of literature on relevant theoretical
            frameworks that examine security within the constraints of FDE, e.g.
            efficient constructions that achieve security in context and subject
            to practical constraints (Rogaway, Fruhwirth, et al).
            \item Outline well-defined cryptographic goals StrongBox should
            achieve; outline guarantees any FDE scheme is expected to provide.
        \end{itemize}
    \item \textbf{6-12/2019}
        \begin{itemize}
            \item Introduce theoretical framework to formalize important
            security notions and evaluate StrongBox FDE, perhaps under IND-CPA,
            IND-CCA, ``time-to-bruteforce'' where relevant, or related
            definitions.
            \item Explore AES-XTS security guarantees versus StrongBox ciphers
            in an appropriate mode, perhaps including key management concerns
            (e.g. circular encryption) and StrongBox's inherent cryptographic
            agility.
        \end{itemize}
    \item \textbf{1-2/2020}
        \begin{itemize}
            \item Collate research into paper: \textit{A Formal Analysis of
            StrongBox FDE}.
        \end{itemize}
\end{itemize}

\renewcommand\thechapter{A}
\chapter{Code Availability} \label{app:availability}

This appendix provides links for the StrongBox I and StrongBox II source code.
The two libraries/APIs that our StrongBox implementations rely on are
additionally provided. Note that resource locations, URLs, frameworks,
interfaces, etc. will likely change overtime, while this text remains static.

You can find instructions on how to build, test, and modify the StrongBox source
in the provided README documentation linked below.

\begin{table}
    \centering
    \caption{Provides URLs for the products yielded and/or used by this research.}
    \begin{tabular}{|r|c|l|l}
        \cline{1-3}
        \textbf{Project Source} & \textbf{Language} & \textbf{URL} & \\
        \cline{1-3}
        StrongBox I Source & C & https://github.com/research/buselfs-public & \\
        StrongBox II Source & C & https://github.com/research/buselfs2-public & \\
        SBD Merkle Tree Implementation & C & https://github.com/IAIK/secure-block-device & \\
        Block Device in User Space (BUSE) & C/Shell & https://github.com/acozzette/BUSE & \\
        \cline{1-3}
    \end{tabular}
\end{table}



% Figures and tables, if you decide to leave them to the end
%\input{figure}
%\input{table}

\clearpage
\printbibliography

\end{document}
