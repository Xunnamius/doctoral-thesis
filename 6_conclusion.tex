\chapter{Conclusion} \label{chp:conclusion}

\TODO{Merge these.}

The conventional wisdom is that securing data at rest requires one must pay the
high performance overhead of encryption with AES is XTS mode. This paper shows
that technological trends overturn this conventional wisdom: Log-structured file
systems and hardware support for secure counters make it practical to use a
stream cipher to secure data at rest. We demonstrate this practicality through
our implementation of StrongBox which uses the ChaCha20 stream cipher and the
Poly1305 MAC to provide secure storage and can be used as a drop-in replacement
for dm-crypt.

Our empirical results show that under F2FS---a modern, industrial-strength
Log-structured file system---StrongBox provides upwards of $2\times$ improvement
on read performance and average $1.27\times$ improvement on write performance
compared to dm-crypt. Further, our results show that F2FS plus StrongBox
provides a higher performance replacement for Ext4 backed with dm-crypt. We make
our implementation of StrongBox available open source so that others can extend
it or compare to it.\footnote{\StrongBoxURI} Our hope is that this work
motivates further exploration of fast stream ciphers as replacements for AES-XTS
for securing data at rest.





This paper advocates for a more agile approach to FDE where the storage system
can dynamically alter the tradeoffs between security and latency/energy. To
support this vision of agile encryption, we have proposed an interface that
allows multiple stream ciphers with different input and output characteristics
to be composed in a generic manner. We have identified three strategies for
using this interface to switch ciphers dynamically, but with low overhead. We
have also proposed a scoring method for determining when to use one cipher over
another. Our case studies show how different strategies can be used to achieve
different goals in practice. We believe that agile encryption will become
increasingly important as successful systems are increasingly required to
balance conflicting operational requirements. We hope that this work inspires
further research into achieving this balance. It is publicly available
open-source\footnoteref{note1}.





Downloading resources over the internet is indeed a risky endeavor. Resource
integrity and other Supply Chain Attacks are becoming more frequent and their
impact more widely felt. In this work, we showed that the de facto standard for
protecting the integrity of arbitrary resources on the internet---the use of
\emph{checksums}---is insufficient and often ineffective. We presented
HASCHK, a practical resource verification protocol that automates the tedious
parts of checksum verification while leveraging pre-existing high availability
systems to ensure resources and their checksums are not vulnerable to
co-hosting. Further, we demonstrated the effectiveness and practicality of our
approach versus real-world resource integrity attacks in a production
application.

The results of our evaluation show that our approach is more effective than
checksums and prior work mitigating integrity attacks for arbitrary resources on
the internet. Further, we show HASCHK is capable of guarding against a
variety of attacks, is deployable at scale for providers that already maintain a
DNS presence, and can be deployed without fear of adversely affecting the user
experience of clients that are not HASCHK-aware.

Though not a panacea, we believe our protocol significantly raises the bar for
the attacker. We intend to continue developing our extension and we make it
available to a wide audience (see \cref{app:availability}).

\section{Future Work}

\TODO{Multithreaded SwitchCrypt explanation goes here.}
