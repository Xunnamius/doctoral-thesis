Security is an important concern among several in modern computer systems. These
concerns often exist in contention over a set of finite resources, e.g. DRAM
bandwidth/capacity, CPU time, drive space. These create a space of possible
tradeoffs that we can optimize against. However, exploiting the slack these
spaces expose in the context of full-drive encryption (FDE) and file systems is
not straightforward.

Full-drive encryption is especially important for mobile devices because they
contain large quantities of sensitive data yet are easily lost or stolen. As
this research demonstrates, the standard approach to FDE—the AES block cipher in
XTS mode—is 3-5x slower than unencrypted storage. Authenticated encryption based
on stream ciphers is already used as a faster alternative to AES in other
contexts, such as HTTPS, but the conventional wisdom is that stream ciphers are
unsuitable for FDE. Used naively in drive encryption, stream ciphers are
vulnerable to attacks, and mitigating these attacks with on-drive metadata is
generally believed to ruin performance.

With this research, we first demonstrate that recent developments in mobile
hardware invalidate this assumption, making it possible to use fast stream
ciphers for FDE. Next, we evaluate a new StrongBox implementation capable of
using ciphers beyond ChaCha20 and AES-CTR to capitalize on security,
performance, and energy tradeoffs. Finally, we will present a theoretical
framework formally analyzing the security of the StrongBox FDE construction.
