\abstract
The security of data at rest---widely understood as FDE or Full Drive
Encryption---is an important concern among several in modern computer systems.
These concerns exist in contention over a set of finite resources. For instance:
a device that is battery-constrained must remain within its energy budget which
may change over time, e.g. when a device enters ``battery-saver mode'';
regardless, this device must meet certain performance guarantees or the user
experience will suffer; above all, the data on the device must be secure from
adversaries; and the device has a finite amount of drive space available. At any
given moment we trade battery life for performance, performance for security,
security for drive space, and so on. Unfortunately, designing a FDE system that
can navigate such treacherous tradeoffs efficiently, effectively, and with
respect to performance and security guarantees is entirely non-trivial. This
dissertation explores this space of tradeoffs and how we might optimize for one
concern without violating another given kernel and/or user space in-context
invariants that might shift over time.

Full drive encryption is especially important for mobile devices because they
contain large quantities of sensitive data yet are easily lost or stolen. As
this research demonstrates, the standard approach to FDE---the AES block cipher
in XTS mode---is 3-5x slower than unencrypted storage. Authenticated encryption
based on stream ciphers is already used as a faster alternative to AES in other
contexts, such as HTTPS, but the conventional wisdom is that stream ciphers are
unsuitable for FDE. Used naively in drive encryption, stream ciphers are
vulnerable to attacks, and mitigating these attacks with on-drive metadata is
generally believed to ruin performance.

We address the difficulty of using stream ciphers for FDE with StrongBox, a
stream cipher based FDE layer that is a drop-in replacement for dm-crypt, the
standard Linux FDE module based on AES-XTS. StrongBox introduces a system design
and on-drive data structures that exploit certain properties of filesystems to
avoid costly rekeying penalties and a counter stored in trusted hardware to
protect against attacks. We implement StrongBox and SwitchCrypt on an ARM
big.LITTLE mobile processor and test its performance under multiple popular
production filesystems.

We push the envelope further with SwitchCrypt, a software mechanism that allows
us to move beyond merely making stream ciphers available for FDE. SwitchCrypt
enables practical navigation of the tradeoff space made by balancing competing
security and latency requirements via \emph{cipher switching} in space or time.
Our key insight in achieving low-overhead switching is to leverage the
overwrite-averse, append-mostly behavior of underlying solid-state storage to
trade throughput for reduced energy use and/or certain security properties.
Similar to StrongBox, we implement SwitchCrypt on an ARM big.LITTLE mobile
processor and test its performance under the popular F2FS LFS. We provide
empirical results demonstrating the conditions under which different switching
strategies are optimal through the exploration of four cases studies.

Finally, with HASCHK, we consider the same stream cipher based cryptographic
primitives in an alternative domain: data \emph{in motion} rather than at rest.
Specifically: securing data downloads over the internet. Such downloads come
with many risks, including the chance that the resource has been corrupted, or
that an attacker has replaced your desired resource with a compromised version.
The de facto standard for addressing this risk is the use of \emph{checksums}
coupled with a secure transport layer; users download a resource, compute its
checksum, and compare that with an authoritative checksum. Problems with this
approach include (1) \emph{user apathy}---for most users, calculating and
verifying the checksum is too tedious; and (2) \emph{co-hosting}---an attacker
who compromises a resource can trivially compromise a checksum hosted on the
same system. The co-hosting problem remains despite advancements in tools that
automate checksum verification and generation. In this dissertation we propose
\emph{HASCHK}, a resource verification protocol expanding on de facto
checksum-based integrity protections to defeat co-hosting while automating the
tedious parts of checksum verification to secure ``data in motion'' over the
internet.

StrongBox, SwitchCrypt, and HASCHK together demonstrate that security is indeed
a paramount concern and valid dimension with which to trade off alongside other
tier-one concerns without compromising data security or requiring obscene
performance sacrifices, all while staying within a shifting energy budget.
