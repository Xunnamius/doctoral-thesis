\section{Evaluation} \label{sec:sb-evaluation}

\subsection{Experimental Setup}

We implement a prototype of StrongBox on a Hardkernel Odroid XU3 ARM big.LITTLE
system (Samsung Exynos5422 A15 and A7 quad core CPUs, 2Gbyte LPDDR3 RAM at 933
MHz, eMMC5.0 HS400 backing store) running Ubuntu Trusty 14.04 LTS, kernel
version 3.10.58. The maximum theoretical memory bandwidth for this model is
14.9GB/s\@. Observed maximum memory bandwidth is 4.5GB/s.

\subsection{Experimental Methodology}

In this section we seek to answer three questions:
\begin{enumerate}
\item What is StrongBox's overhead when compared to dm-crypt AES-XTS?
\item How does StrongBox under an LFS (\ie F2FS) configuration compare to
the popular dm-crypt under Ext4 configuration?
\item Where does StrongBox derive its performance gains? Implementation? Choice
of cipher?
%\item How does StrongBox perform when the backing store is nearly full?
\end{enumerate}

To evaluate StrongBox's performance, we measure the latency
(seconds/milliseconds per operation) of both sequential and random
read and write I/O operations across four different standard Linux
filesystems: NILFS2, F2FS, Ext4 in ordered journaling mode, and Ext4
in full journaling mode. The I/O operations are performed using file
sizes between 4KB and 40MB. These files were populated with random
data. The experiments are performed using a 1GB standard Linux ramdisk
(tmpfs) as the ultimate backing store.

For sequential F2FS specifically, we include latency measurements dealing
with a file size $2.5\times$ the size of available DRAM, \ie
5GB, supported by a distinct tmpfs backing store swapped into memory.

Ext4's default is ordered journaling mode (\texttt{data=ordered}),
where metadata is committed to the filesystem's journal while the
actual data is written through to the main filesystem. Given a crash,
the filesystem uses the journal to avoid damage and recover to a
consistent state. Full journaling mode (\texttt{data=journal})
journals both metadata and the filesystem's actual data---essentially
a double write-back for each write operation. Given a crash, the
journal can replay entire I/O events so that both the filesystem and
its data can be recovered. We include both modes of Ext4 to further
explore the impact of frequent overwrites against StrongBox.

The experiment consists of reading and writing each file in its
entirety 30 times sequentially, and then reading and writing random
portions of each file 30 times. In both cases, the same amount of data
is read and written per file. The median latency is taken per result
set. We chose 30 read/write operations (10 read/write operations
repeated three times each) to handle potential variation. The Linux
page cache is dropped before every read operation, each file is
opened in synchronous I/O mode via \texttt{O\_SYNC}, and we rely on
non-buffered \texttt{read()}/\texttt{write()} system calls. A
high-level I/O size of 128KB was used for all read and write calls
that hit the filesystems; however, the I/O requests being made at the
block device layer varied between 4KB and 128KB depending on the
filesystem under test.

The experiment is repeated on each filesystem in three different
configurations:
\begin{enumerate}
\item \textit{unencrypted}: Filesystem mounted atop a BUSE virtual block
  device set up to immediately pass through any incoming I/O requests straight
  to the backing store. We use this as the baseline measurement of the
  filesystem's performance without any encryption.
\item \textit{StrongBox}: Filesystem mounted atop a BUSE virtual block
  device provided by our StrongBox implementation to perform full drive
  encryption.
\item \textit{dm-crypt}: Filesystem mounted atop a Device Mapper
 ~\cite{LinuxDeviceMapper} higher-level virtual block device provided by
  dm-crypt to perform full drive encryption, which itself is mounted atop a
  BUSE virtual block device with pass through behavior identical to the device
  used in the baseline configuration. dm-crypt was configured to use AES-XTS as
  its full drive encryption algorithm. All other parameters were left at their
  default values.
\end{enumerate}

\figref{microbench-f2fs} compares StrongBox to dm-crypt under the F2FS
filesystem. The gamut of result sets over different filesystems can
be seen in \figref{microbench-gamut}. \figref{microbench-ext4}
compares Ext4 with dm-crypt to F2FS with StrongBox.

\begin{figure}[ht]
    \textbf{StrongBox vs dm-crypt AES-XTS: F2FS Test}\par\medskip
    \begin{subfigure}{\linewidth}
        \centering
        {\begin{tikzpicture}[baseline]

% XXX: update 0
\pgfmathsetmacro{\ymax}{4} % set the maximum y value
\pgfmathsetmacro{\ymaxbreak}{4.075} % set the y value at which overflow is drawn

\begin{groupplot}[
    group style={
        group name=plots,
        group size=1 by 1,
        xlabels at=edge top,
        xticklabels at=edge top,
        vertical sep=5pt
    },
    axis x line*=top,
    xlabel near ticks,
    major x tick style=transparent,
    height=6cm,
    width=\linewidth,
    xmin=0, xmax=5,
    % enlarge y limits={value=0.2,upper},
    tick align=outside,
    tick style={white},
    ytick=\empty,
    xtick=\empty,
    xticklabels={},
    yticklabels={},
    % restrict y to domain*=0:2,
    clip=false,
]
\nextgroupplot[
    ylabel={\footnotesize Latency (normalized to unencrypted)}, % XXX: update 1
    ylabel shift={6mm},
    ymin=0, ymax=1,
]
\end{groupplot}

\begin{groupplot}[
    group style={
        group name=plots,
        group size=1 by 1,
        xlabels at=edge bottom,
        xticklabels at=edge bottom,
        vertical sep=5pt
    },
    axis x line*=bottom,
    xlabel near ticks,
    major x tick style=transparent,
    height=6cm,
    width=\linewidth,
    %xmin=0, %xmax=2,
    xlabel={\footnotesize File Size (bytes)},
    symbolic x coords={4K,512K,5M,40M,5G,Mean},
    xtick=data,
    enlarge x limits=0.2, % add some breathing room along the x axis's sides
    extra y ticks={1},
    extra y tick style={grid=major, grid style={dashed, black}},
    extra y tick label={ 1 },
    tick align=outside,
    tick style={ black },
    legend cell align=center,
    legend style={ column sep=1ex },
    ymajorgrids=true,
    grid style={ dotted, gray },
    every node near coord/.append style={font=\tiny},
    % magic to make the numbers appear above the overly long bars:
    visualization depends on={rawy \as \rawy}, % save original y values
    restrict y to domain*={ % now clip/restrict any y value to ymax
        \pgfkeysvalueof{/pgfplots/ymin}:\ymaxbreak
    },
    after end axis/.code={ % draw squiggly line indicating break
        \draw [semithick, white, decoration={snake,amplitude=0.1mm,segment length=0.75mm,post length=0.375mm}, decorate] (rel axis cs:0,1.01) -- (rel axis cs:1,1.01);
    },
    nodes near coords={\color{.!75!black}\pgfmathprintnumber\rawy}, % print the original y values (darkened in case they're too light)...
    nodes near coords greater equal only=\ymax, % ... but ONLY if they're >= ymax
    clip=false, % allow clip to protrude beyond ymax
    %
    % Custom stuff to edit per template
    % XXX: update 2
    ymin=1, ymax=\ymax,
    ytick={ 1, 2, 3, 4 },
    yticklabels={ 1, 2, 3, 4 },
]
\nextgroupplot[
    % XXX: update 3
    ybar=1pt, % change space between bars
    bar width=4.5pt, % change size of bars
    legend entries={
        {\scriptsize StrongBox/reads},
        {\scriptsize dm-crypt/reads},
        {\scriptsize StrongBox/writes},
        {\scriptsize dm-crypt/writes},
    },
    legend style={
        draw=none,
        legend columns=2,
        at={(0.525,1.3)},
        anchor=north
    },
    clip=false
]

% XXX: update 4
\addplot[fill=orangeLight, every node near coord/.append style={color=orangeLight}]
table[x=size, y=sR, col sep=space] {data/sb/microbench-f2fs-sequential.dat};
\addplot[fill=purpleDark, every node near coord/.append style={color=purpleDark}]
table[x=size, y=dR, col sep=space] {data/sb/microbench-f2fs-sequential.dat};
\addplot[draw=orangeLight, pattern=crosshatch, pattern color=orangeLight, every node near coord/.append style={color=orangeLight}]
table[x=size, y=sW, col sep=space] {data/sb/microbench-f2fs-sequential.dat};
\addplot[draw=purpleDark, pattern=crosshatch, pattern color=purpleDark, every node near coord/.append style={color=purpleDark}]
table[x=size, y=dW, col sep=space] {data/sb/microbench-f2fs-sequential.dat};

\end{groupplot}%
\end{tikzpicture}%
}
        \caption{Sequential I/O expanded F2FS result set.} \label{fig:microbench-f2fs-sequential}
    \end{subfigure}\\[1ex]
    \begin{subfigure}{\linewidth}
        \centering
        {\begin{tikzpicture}[baseline]

% XXX: update 0
\pgfmathsetmacro{\ymax}{4} % set the maximum y value
\pgfmathsetmacro{\ymaxbreak}{4.075} % set the y value at which overflow is drawn

\begin{groupplot}[
    group style={
        group name=plots,
        group size=1 by 1,
        xlabels at=edge top,
        xticklabels at=edge top,
        vertical sep=5pt
    },
    axis x line*=top,
    xlabel near ticks,
    major x tick style=transparent,
    height=6cm,
    width=\linewidth,
    xmin=0, xmax=5,
    % enlarge y limits={value=0.2,upper},
    tick align=outside,
    tick style={white},
    ytick=\empty,
    xtick=\empty,
    xticklabels={},
    yticklabels={},
    % restrict y to domain*=0:2,
    clip=false,
]
\nextgroupplot[
    ylabel={\footnotesize Latency (normalized to unencrypted)}, % XXX: update 1
    ylabel shift={6mm},
    ymin=0, ymax=1,
]
\end{groupplot}

\begin{groupplot}[
    group style={
        group name=plots,
        group size=1 by 1,
        xlabels at=edge bottom,
        xticklabels at=edge bottom,
        vertical sep=5pt
    },
    axis x line*=bottom,
    xlabel near ticks,
    major x tick style=transparent,
    height=6cm,
    width=\linewidth,
    %xmin=0, %xmax=2,
    xlabel={\footnotesize File Size (bytes)},
    symbolic x coords={4K,512K,5M,40M,Mean},
    xtick=data,
    enlarge x limits=0.2, % add some breathing room along the x axis's sides
    extra y ticks={1},
    extra y tick style={grid=major, grid style={dashed, black}},
    extra y tick label={ 1 },
    tick align=outside,
    tick style={ black },
    legend cell align=center,
    legend style={ column sep=1ex },
    ymajorgrids=true,
    grid style={ dotted, gray },
    every node near coord/.append style={font=\tiny},
    % magic to make the numbers appear above the overly long bars:
    visualization depends on={rawy \as \rawy}, % save original y values
    restrict y to domain*={ % now clip/restrict any y value to ymax
        \pgfkeysvalueof{/pgfplots/ymin}:\ymaxbreak
    },
    after end axis/.code={ % draw squiggly line indicating break
        \draw [semithick, white, decoration={snake,amplitude=0.1mm,segment length=0.75mm,post length=0.375mm}, decorate] (rel axis cs:0,1.01) -- (rel axis cs:1,1.01);
    },
    nodes near coords={\color{.!75!black}\pgfmathprintnumber\rawy}, % print the original y values (darkened in case they're too light)...
    nodes near coords greater equal only=\ymax, % ... but ONLY if they're >= ymax
    clip=false, % allow clip to protrude beyond ymax
    %
    % Custom stuff to edit per template
    % XXX: update 2
    ymin=1, ymax=\ymax,
    ytick={ 1, 2, 3, 4 },
    yticklabels={ 1, 2, 3, 4 },
]
\nextgroupplot[
    % XXX: update 3
    ybar=1pt, % change space between bars
    bar width=4.5pt, % change size of bars
    clip=false
]

% XXX: update 4
\addplot[fill=orangeLight, every node near coord/.append style={color=orangeLight}]
table[x=size, y=sR, col sep=space] {img/microbench-f2fs-random.dat};
\addplot[fill=purpleDark, every node near coord/.append style={color=purpleDark}]
table[x=size, y=dR, col sep=space] {img/microbench-f2fs-random.dat};
\addplot[draw=orangeLight, pattern=crosshatch, pattern color=orangeLight, every node near coord/.append style={color=orangeLight}]
table[x=size, y=sW, col sep=space] {img/microbench-f2fs-random.dat};
\addplot[draw=purpleDark, pattern=crosshatch, pattern color=purpleDark, every node near coord/.append style={color=purpleDark}]
table[x=size, y=dW, col sep=space] {img/microbench-f2fs-random.dat};

\end{groupplot}%
\end{tikzpicture}%
}
        \caption{Random I/O expanded F2FS result set.} \label{fig:microbench-f2fs-random}
    \end{subfigure}
    \caption{Test of the F2FS LFS mounted atop both dm-crypt and
      StrongBox; median latency of different sized whole file read and
      write operations normalized to unencrypted access. By harmonic
      mean, StrongBox is $1.72\times$ faster than dm-crypt for sequential reads
      and $1.27\times$ faster for sequential writes.} \label{fig:microbench-f2fs}
\end{figure}

\subsection{StrongBox Read Performance}

\figref{microbench-f2fs} shows the performance of StrongBox in comparison to
dm-crypt, both mounted with the F2FS filesystem. We see StrongBox improves on
the performance of dm-crypt's AES-XTS implementation across sequential and
random read operations on all file sizes. Specifically, $2.36\times$
(53.05m/22.48m) for sequential 5GB, $2.07\times$ (2.09s/1.00s) for sequential
40MB, $2.08\times$ (267.34ms/128.22ms) for sequential 5MB, $1.85\times$
(28.30ms/15.33ms) for sequential 512KB, and $1.03\times$ (0.95ms/0.86ms) for
sequential 4KB\@.

\figref{microbench-gamut} provides an expanded performance profile for
StrongBox, testing a gamut of filesystems broken down by workload file size. For
sequential reads across all filesystems and file sizes, StrongBox outperforms
dm-crypt. This is true even on the non-LFS Ext4 filesystems. Specifically, we
see read performance improvements over dm-crypt AES-XTS for 40MB sequential
reads of $2.02\times$ (2.15s/1.06s) for NILFS, $2.07\times$ (2.09s/1.00s) for
F2FS, $2.09\times$ (2.11s/1.01s) for Ext4 in ordered journaling mode, and
$2.06\times$ (2.11s/1.02s) for Ext4 in full journaling mode. For smaller file
sizes, the performance improvement is less pronounced. For 4KB reads we see
$1.28\times$ (1.62ms/1.26ms) for NILFS, $1.03\times$ (0.88ms/0.86ms) for F2FS,
$1.04\times$ (0.95ms/0.92ms) for Ext4 in ordered journaling mode, and
$1.07\times$ (0.97ms/0.91ms) for Ext4 in full journaling mode. When it comes to
random reads, we see virtually identical results save for 4KB reads, where
dm-crypt proved slightly more performant under the NILFS LFS at $1.12\times$
(1.73ms/1.54ms). This behavior is not observed with the more modern F2FS\@.

% \begin{figure*}[t]
%     \textbf{StrongBox Four Filesystems Test}\par\medskip
%     \centering
%     \begin{subfigure}{0.5\linewidth}
%         \centering
%         {\begin{tikzpicture}[baseline]

% XXX: update 0
\pgfmathsetmacro{\ymax}{2} % set the maximum y value
\pgfmathsetmacro{\ymaxbreak}{2.05} % set the y value at which overflow is drawn

\begin{groupplot}[
    group style={
        group name=plots,
        group size=1 by 1,
        xlabels at=edge top,
        xticklabels at=edge top,
        vertical sep=5pt,
    },
    axis x line*=top,
    xlabel near ticks,
    major x tick style=transparent,
    height=6cm,
    width=\linewidth,
    xmin=0, xmax=5,
    % enlarge y limits={value=0.2,upper},
    tick align=outside,
    tick style={white},
    ytick=\empty,
    xtick=\empty,
    xticklabels={},
    yticklabels={},
    % restrict y to domain*=0:2,
    clip=false,
]
\nextgroupplot[
    ylabel={\footnotesize Latency (normalized to dm-crypt)}, % XXX: update 1
    ylabel shift={6mm},
    ymin=0, ymax=1,
]
\end{groupplot}

\begin{groupplot}[
    group style={
        group name=plots,
        group size=1 by 1,
        xlabels at=edge bottom,
        xticklabels at=edge bottom,
        vertical sep=5pt
    },
    axis x line*=bottom,
    xlabel near ticks,
    major x tick style=transparent,
    height=6cm,
    width=\linewidth,
    %xmin=0, %xmax=2,
    xlabel={\footnotesize File Size (bytes)},
    symbolic x coords={4K,512K,5M,40M},
    xticklabels={\footnotesize{4K}, \footnotesize{512K}, \footnotesize{5M}, \footnotesize{40M}},
    xtick=data,
    enlarge x limits=0.2, % add some breathing room along the x axis's sides
    extra y ticks={1},
    extra y tick style={grid=major, grid style={dashed, black}},
    extra y tick label={ 1 },
    tick align=outside,
    tick style={ black },
    legend cell align=center,
    legend style={ column sep=1ex },
    ymajorgrids=true,
    grid style={ dotted, gray },
    every node near coord/.append style={font=\tiny},
    % magic to make the numbers appear above the overly long bars:
    visualization depends on={rawy \as \rawy}, % save original y values
    restrict y to domain*={ % now clip/restrict any y value to ymax
        \pgfkeysvalueof{/pgfplots/ymin}:\ymaxbreak
    },
    after end axis/.code={ % draw squiggly line indicating break
        \draw [semithick, white, decoration={snake,amplitude=0.1mm,segment length=0.75mm,post length=0.375mm}, decorate] (rel axis cs:0,1.01) -- (rel axis cs:1,1.01);
    },
    nodes near coords={\color{.!75!black}\pgfmathprintnumber\rawy}, % print the original y values (darkened in case they're too light)...
    nodes near coords greater equal only=\ymax, % ... but ONLY if they're >= ymax
    clip=false, % allow clip to protrude beyond ymax
    %
    % Custom stuff to edit per template
    % XXX: update 2
    ymin=0, ymax=\ymax,
    ytick={ 0, 0.5, 1.5, 2 },
    yticklabels={ 0, 0.5, 1.5, 2 },
]
\nextgroupplot[
    % XXX: update 3
    ybar=1pt, % change space between bars
    bar width=2.25pt,
    legend entries={
        {\scriptsize NILFS/reads},
        {\scriptsize F2FS/reads},
        {\scriptsize Ext4OJ/reads},
        {\scriptsize Ext4FJ/reads},
    },
    legend style={
        draw=none,
        legend columns=2,
        at={(0.35,1.35)},
        anchor=north
    },
    clip=false
]

% XXX: update 4
\addplot[fill=orangeDark, every node near coord/.append style={color=orangeDark}]
table[x=size, y=NILFS, col sep=space] {data/sb/microbench-gamut-sequential-r.dat};
\addplot[fill=purpleDark, every node near coord/.append style={color=purpleDark}]
table[x=size, y=F2FS, col sep=space] {data/sb/microbench-gamut-sequential-r.dat};
\addplot[draw=orangeLight, pattern=crosshatch, pattern color=orangeLight, every node near coord/.append style={color=orangeLight}]
table[x=size, y=Ext4OJ, col sep=space] {data/sb/microbench-gamut-sequential-r.dat};
\addplot[draw=purpleLight, pattern=crosshatch, pattern color=purpleDark, every node near coord/.append style={color=purpleDark}]
table[x=size, y=Ext4FJ, col sep=space] {data/sb/microbench-gamut-sequential-r.dat};

\end{groupplot}%
\end{tikzpicture}%
}
%         \caption{Sequential reads.}
%        \label{fig:microbench-gamut-sequential-r}
%     \end{subfigure}\hspace*{0.5em}%
%     \begin{subfigure}{0.5\linewidth}
%         \centering
%         {\begin{tikzpicture}[baseline]

% XXX: update 0
\pgfmathsetmacro{\ymax}{2} % set the maximum y value
\pgfmathsetmacro{\ymaxbreak}{2.04} % set the y value at which overflow is drawn

\begin{groupplot}[
    group style={
        group name=plots,
        group size=1 by 1,
        xlabels at=edge top,
        xticklabels at=edge top,
        vertical sep=5pt,
    },
    axis x line*=top,
    xlabel near ticks,
    major x tick style=transparent,
    height=6cm,
    width=\linewidth,
    xmin=0, xmax=5,
    % enlarge y limits={value=0.2,upper},
    tick align=outside,
    tick style={white},
    ytick=\empty,
    xtick=\empty,
    xticklabels={},
    yticklabels={},
    % restrict y to domain*=0:2,
    clip=false,
]
\nextgroupplot[
    ylabel={}, % XXX: update 1
    ylabel shift={6mm},
    ymin=0, ymax=1,
]
\end{groupplot}

\begin{groupplot}[
    group style={
        group name=plots,
        group size=1 by 1,
        xlabels at=edge bottom,
        xticklabels at=edge bottom,
        vertical sep=5pt
    },
    axis x line*=bottom,
    xlabel near ticks,
    major x tick style=transparent,
    height=6cm,
    width=\linewidth,
    %xmin=0, %xmax=2,
    xlabel={\footnotesize File Size (bytes)},
    symbolic x coords={4K,512K,5M,40M},
    xticklabels={\footnotesize{4K}, \footnotesize{512K}, \footnotesize{5M}, \footnotesize{40M}},
    xtick=data,
    enlarge x limits=0.2, % add some breathing room along the x axis's sides
    extra y ticks={1},
    extra y tick style={grid=major, grid style={dashed, black}},
    extra y tick label={ 1 },
    tick align=outside,
    tick style={ black },
    legend cell align=center,
    legend style={ column sep=1ex },
    ymajorgrids=true,
    grid style={ dotted, gray },
    every node near coord/.append style={font=\tiny},
    % magic to make the numbers appear above the overly long bars:
    visualization depends on={rawy \as \rawy}, % save original y values
    restrict y to domain*={ % now clip/restrict any y value to ymax
        \pgfkeysvalueof{/pgfplots/ymin}:\ymaxbreak
    },
    after end axis/.code={ % draw squiggly line indicating break
        \draw [semithick, white, decoration={snake,amplitude=0.1mm,segment length=0.75mm,post length=0.375mm}, decorate] (rel axis cs:0,1.01) -- (rel axis cs:1,1.01);
    },
    nodes near coords={\color{.!75!black}\pgfmathprintnumber\rawy}, % print the original y values (darkened in case they're too light)...
    nodes near coords greater equal only=\ymax, % ... but ONLY if they're >= ymax
    clip=false, % allow clip to protrude beyond ymax
    %
    % Custom stuff to edit per template
    % XXX: update 2
    ymin=0, ymax=\ymax,
    ytick={ 0, 0.5, 1.5, 2 },
    yticklabels={ 0, 0.5, 1.5, 2 },
]
\nextgroupplot[
    % XXX: update 3
    ybar=1pt, % change space between bars
    bar width=2.25pt,
    legend entries={
        {\scriptsize NILFS/writes},
        {\scriptsize F2FS/writes},
        {\scriptsize Ext4OJ/writes},
        {\scriptsize Ext4FJ/writes},
    },
    legend style={
        draw=none,
        legend columns=2,
        at={(0.5,1.35)},
        anchor=north
    },
    clip=false
]

% XXX: update 4
\addplot[fill=orangeDark, every node near coord/.append style={color=orangeDark}]
table[x=size, y=NILFS, col sep=space] {data/sb/microbench-gamut-sequential-w.dat};
\addplot[fill=purpleDark, every node near coord/.append style={color=purpleDark}]
table[x=size, y=F2FS, col sep=space] {data/sb/microbench-gamut-sequential-w.dat};
\addplot[draw=orangeLight, pattern=crosshatch, pattern color=orangeLight, every node near coord/.append style={color=orangeLight, yshift=0.15cm}]
table[x=size, y=Ext4OJ, col sep=space] {data/sb/microbench-gamut-sequential-w.dat};
\addplot[draw=purpleLight, pattern=crosshatch, pattern color=purpleDark, every node near coord/.append style={color=purpleDark}]
table[x=size, y=Ext4FJ, col sep=space] {data/sb/microbench-gamut-sequential-w.dat};

\end{groupplot}%
\end{tikzpicture}%
}
%         \caption{Sequential writes.}
%        \label{fig:microbench-gamut-sequential-w}
%     \end{subfigure}\\[1ex]
%     \hspace*{-0.9em}%
%     \begin{subfigure}{0.5\linewidth}
%         \vspace{0.5em}
%         \centering
%         {\begin{tikzpicture}[baseline]

% XXX: update 0
\pgfmathsetmacro{\ymax}{2} % set the maximum y value
\pgfmathsetmacro{\ymaxbreak}{2.05} % set the y value at which overflow is drawn

\begin{groupplot}[
    group style={
        group name=plots,
        group size=1 by 1,
        xlabels at=edge top,
        xticklabels at=edge top,
        vertical sep=5pt,
    },
    axis x line*=top,
    xlabel near ticks,
    major x tick style=transparent,
    height=6cm,
    width=\linewidth,
    xmin=0, xmax=5,
    % enlarge y limits={value=0.2,upper},
    tick align=outside,
    tick style={white},
    ytick=\empty,
    xtick=\empty,
    xticklabels={},
    yticklabels={},
    % restrict y to domain*=0:2,
    clip=false,
]
\nextgroupplot[
    ylabel={\footnotesize Latency (normalized to dm-crypt)}, % XXX: update 1
    ylabel shift={6mm},
    ymin=0, ymax=1,
]
\end{groupplot}

\begin{groupplot}[
    group style={
        group name=plots,
        group size=1 by 1,
        xlabels at=edge bottom,
        xticklabels at=edge bottom,
        vertical sep=5pt
    },
    axis x line*=bottom,
    xlabel near ticks,
    major x tick style=transparent,
    height=6cm,
    width=\linewidth,
    %xmin=0, %xmax=2,
    xlabel={\footnotesize File Size (bytes)},
    symbolic x coords={4K,512K,5M,40M},
    xticklabels={\footnotesize{4K}, \footnotesize{512K}, \footnotesize{5M}, \footnotesize{40M}},
    xtick=data,
    enlarge x limits=0.2, % add some breathing room along the x axis's sides
    extra y ticks={1},
    extra y tick style={grid=major, grid style={dashed, black}},
    extra y tick label={ 1 },
    tick align=outside,
    tick style={ black },
    legend cell align=center,
    legend style={ column sep=1ex },
    ymajorgrids=true,
    grid style={ dotted, gray },
    every node near coord/.append style={font=\tiny},
    % magic to make the numbers appear above the overly long bars:
    visualization depends on={rawy \as \rawy}, % save original y values
    restrict y to domain*={ % now clip/restrict any y value to ymax
        \pgfkeysvalueof{/pgfplots/ymin}:\ymaxbreak
    },
    after end axis/.code={ % draw squiggly line indicating break
        \draw [semithick, white, decoration={snake,amplitude=0.1mm,segment length=0.75mm,post length=0.375mm}, decorate] (rel axis cs:0,1.01) -- (rel axis cs:1,1.01);
    },
    nodes near coords={\color{.!75!black}\pgfmathprintnumber\rawy}, % print the original y values (darkened in case they're too light)...
    nodes near coords greater equal only=\ymax, % ... but ONLY if they're >= ymax
    clip=false, % allow clip to protrude beyond ymax
    %
    % Custom stuff to edit per template
    % XXX: update 2
    ymin=0, ymax=\ymax,
    ytick={ 0, 0.5, 1.5, 2 },
    yticklabels={ 0, 0.5, 1.5, 2 },
]
\nextgroupplot[
    % XXX: update 3
    ybar=1pt, % change space between bars
    bar width=2.25pt,
    clip=false
]

% XXX: update 4
\addplot[fill=orangeDark, every node near coord/.append style={color=orangeDark}]
table[x=size, y=NILFS, col sep=space] {img/microbench-gamut-random-r.dat};
\addplot[fill=purpleDark, every node near coord/.append style={color=purpleDark}]
table[x=size, y=F2FS, col sep=space] {img/microbench-gamut-random-r.dat};
\addplot[draw=orangeLight, pattern=crosshatch, pattern color=orangeLight, every node near coord/.append style={color=orangeLight}]
table[x=size, y=Ext4OJ, col sep=space] {img/microbench-gamut-random-r.dat};
\addplot[draw=purpleLight, pattern=crosshatch, pattern color=purpleDark, every node near coord/.append style={color=purpleDark}]
table[x=size, y=Ext4FJ, col sep=space] {img/microbench-gamut-random-r.dat};

\end{groupplot}%
\end{tikzpicture}%
}
%         \caption{Random reads.}
%        \label{fig:microbench-gamut-random-r}
%     \end{subfigure}%
%     \begin{subfigure}{0.5\linewidth}
%         \centering
%         {\begin{tikzpicture}[baseline]

% XXX: update 0
\pgfmathsetmacro{\ymax}{2} % set the maximum y value
\pgfmathsetmacro{\ymaxbreak}{2.04} % set the y value at which overflow is drawn

\begin{groupplot}[
    group style={
        group name=plots,
        group size=1 by 1,
        xlabels at=edge top,
        xticklabels at=edge top,
        vertical sep=5pt,
    },
    axis x line*=top,
    xlabel near ticks,
    major x tick style=transparent,
    height=6cm,
    width=\linewidth,
    xmin=0, xmax=5,
    % enlarge y limits={value=0.2,upper},
    tick align=outside,
    tick style={white},
    ytick=\empty,
    xtick=\empty,
    xticklabels={},
    yticklabels={},
    % restrict y to domain*=0:2,
    clip=false,
]
\nextgroupplot[
    ylabel={}, % XXX: update 1
    ylabel shift={6mm},
    ymin=0, ymax=1,
]
\end{groupplot}

\begin{groupplot}[
    group style={
        group name=plots,
        group size=1 by 1,
        xlabels at=edge bottom,
        xticklabels at=edge bottom,
        vertical sep=5pt
    },
    axis x line*=bottom,
    xlabel near ticks,
    major x tick style=transparent,
    height=6cm,
    width=\linewidth,
    %xmin=0, %xmax=2,
    xlabel={\footnotesize File Size (bytes)},
    symbolic x coords={4K,512K,5M,40M},
    xticklabels={\footnotesize{4K}, \footnotesize{512K}, \footnotesize{5M}, \footnotesize{40M}},
    xtick=data,
    enlarge x limits=0.2, % add some breathing room along the x axis's sides
    extra y ticks={1},
    extra y tick style={grid=major, grid style={dashed, black}},
    extra y tick label={ 1 },
    tick align=outside,
    tick style={ black },
    legend cell align=center,
    legend style={ column sep=1ex },
    ymajorgrids=true,
    grid style={ dotted, gray },
    every node near coord/.append style={font=\tiny},
    % magic to make the numbers appear above the overly long bars:
    visualization depends on={rawy \as \rawy}, % save original y values
    restrict y to domain*={ % now clip/restrict any y value to ymax
        \pgfkeysvalueof{/pgfplots/ymin}:\ymaxbreak
    },
    after end axis/.code={ % draw squiggly line indicating break
        \draw [semithick, white, decoration={snake,amplitude=0.1mm,segment length=0.75mm,post length=0.375mm}, decorate] (rel axis cs:0,1.01) -- (rel axis cs:1,1.01);
    },
    nodes near coords={\color{.!75!black}\pgfmathprintnumber\rawy}, % print the original y values (darkened in case they're too light)...
    nodes near coords greater equal only=\ymax, % ... but ONLY if they're >= ymax
    clip=false, % allow clip to protrude beyond ymax
    %
    % Custom stuff to edit per template
    % XXX: update 2
    ymin=0, ymax=\ymax,
    ytick={ 0, 0.5, 1.5, 2 },
    yticklabels={ 0, 0.5, 1.5, 2 },
]
\nextgroupplot[
    % XXX: update 3
    ybar=1pt, % change space between bars
    bar width=2.25pt,
    clip=false
]

% XXX: update 4
\addplot[fill=orangeDark, every node near coord/.append style={color=orangeDark}]
table[x=size, y=NILFS, col sep=space] {img/microbench-gamut-random-w.dat};
\addplot[fill=purpleDark, every node near coord/.append style={color=purpleDark}]
table[x=size, y=F2FS, col sep=space] {img/microbench-gamut-random-w.dat};
\addplot[draw=orangeLight, pattern=crosshatch, pattern color=orangeLight, every node near coord/.append style={color=orangeLight, yshift=0.15cm}]
table[x=size, y=Ext4OJ, col sep=space] {img/microbench-gamut-random-w.dat};
\addplot[draw=purpleLight, pattern=crosshatch, pattern color=purpleDark, every node near coord/.append style={color=purpleDark}]
table[x=size, y=Ext4FJ, col sep=space] {img/microbench-gamut-random-w.dat};

\end{groupplot}%
\end{tikzpicture}%
}
%         \caption{Random writes.}
%        \label{fig:microbench-gamut-random-w}
%     \end{subfigure}
%     \caption{Comparison of four filesystems running on top of
%       StrongBox performance is normalized to the same file system
%       running on dm-crypt. Points below the line signify StrongBox
%       outperforming dm-crypt. Points above the line signify dm-crypt
%       outperforming StrongBox.}
%    \label{fig:microbench-gamut}
% \end{figure*}

\begin{figure}[ht]
    \textbf{StrongBox Four Filesystems Test}\par\medskip
    \centering
    \begin{subfigure}{0.5\linewidth}
        \centering
        {\begin{tikzpicture}[baseline]

% XXX: update 0
\pgfmathsetmacro{\ymax}{2} % set the maximum y value
\pgfmathsetmacro{\ymaxbreak}{2.05} % set the y value at which overflow is drawn

\begin{groupplot}[
    group style={
        group name=plots,
        group size=1 by 1,
        xlabels at=edge top,
        xticklabels at=edge top,
        vertical sep=5pt,
    },
    axis x line*=top,
    xlabel near ticks,
    major x tick style=transparent,
    height=6cm,
    width=\linewidth,
    xmin=0, xmax=5,
    % enlarge y limits={value=0.2,upper},
    tick align=outside,
    tick style={white},
    ytick=\empty,
    xtick=\empty,
    xticklabels={},
    yticklabels={},
    % restrict y to domain*=0:2,
    clip=false,
]
\nextgroupplot[
    ylabel={\footnotesize Latency (normalized to dm-crypt)}, % XXX: update 1
    ylabel shift={6mm},
    ymin=0, ymax=1,
]
\end{groupplot}

\begin{groupplot}[
    group style={
        group name=plots,
        group size=1 by 1,
        xlabels at=edge bottom,
        xticklabels at=edge bottom,
        vertical sep=5pt
    },
    axis x line*=bottom,
    xlabel near ticks,
    major x tick style=transparent,
    height=6cm,
    width=\linewidth,
    %xmin=0, %xmax=2,
    xlabel={\footnotesize File Size (bytes)},
    symbolic x coords={4K,512K,5M,40M},
    xticklabels={\footnotesize{4K}, \footnotesize{512K}, \footnotesize{5M}, \footnotesize{40M}},
    xtick=data,
    enlarge x limits=0.2, % add some breathing room along the x axis's sides
    extra y ticks={1},
    extra y tick style={grid=major, grid style={dashed, black}},
    extra y tick label={ 1 },
    tick align=outside,
    tick style={ black },
    legend cell align=center,
    legend style={ column sep=1ex },
    ymajorgrids=true,
    grid style={ dotted, gray },
    every node near coord/.append style={font=\tiny},
    % magic to make the numbers appear above the overly long bars:
    visualization depends on={rawy \as \rawy}, % save original y values
    restrict y to domain*={ % now clip/restrict any y value to ymax
        \pgfkeysvalueof{/pgfplots/ymin}:\ymaxbreak
    },
    after end axis/.code={ % draw squiggly line indicating break
        \draw [semithick, white, decoration={snake,amplitude=0.1mm,segment length=0.75mm,post length=0.375mm}, decorate] (rel axis cs:0,1.01) -- (rel axis cs:1,1.01);
    },
    nodes near coords={\color{.!75!black}\pgfmathprintnumber\rawy}, % print the original y values (darkened in case they're too light)...
    nodes near coords greater equal only=\ymax, % ... but ONLY if they're >= ymax
    clip=false, % allow clip to protrude beyond ymax
    %
    % Custom stuff to edit per template
    % XXX: update 2
    ymin=0, ymax=\ymax,
    ytick={ 0, 0.5, 1.5, 2 },
    yticklabels={ 0, 0.5, 1.5, 2 },
]
\nextgroupplot[
    % XXX: update 3
    ybar=1pt, % change space between bars
    bar width=2.25pt,
    legend entries={
        {\scriptsize NILFS/reads},
        {\scriptsize F2FS/reads},
        {\scriptsize Ext4OJ/reads},
        {\scriptsize Ext4FJ/reads},
    },
    legend style={
        draw=none,
        legend columns=2,
        at={(0.35,1.35)},
        anchor=north
    },
    clip=false
]

% XXX: update 4
\addplot[fill=orangeDark, every node near coord/.append style={color=orangeDark}]
table[x=size, y=NILFS, col sep=space] {data/sb/microbench-gamut-sequential-r.dat};
\addplot[fill=purpleDark, every node near coord/.append style={color=purpleDark}]
table[x=size, y=F2FS, col sep=space] {data/sb/microbench-gamut-sequential-r.dat};
\addplot[draw=orangeLight, pattern=crosshatch, pattern color=orangeLight, every node near coord/.append style={color=orangeLight}]
table[x=size, y=Ext4OJ, col sep=space] {data/sb/microbench-gamut-sequential-r.dat};
\addplot[draw=purpleLight, pattern=crosshatch, pattern color=purpleDark, every node near coord/.append style={color=purpleDark}]
table[x=size, y=Ext4FJ, col sep=space] {data/sb/microbench-gamut-sequential-r.dat};

\end{groupplot}%
\end{tikzpicture}%
}
        \caption{Sequential reads.}
       \label{fig:microbench-gamut-sequential-r}
    \end{subfigure}\hspace*{0.5em}%
    \begin{subfigure}{0.5\linewidth}
        \centering
        {\begin{tikzpicture}[baseline]

% XXX: update 0
\pgfmathsetmacro{\ymax}{2} % set the maximum y value
\pgfmathsetmacro{\ymaxbreak}{2.04} % set the y value at which overflow is drawn

\begin{groupplot}[
    group style={
        group name=plots,
        group size=1 by 1,
        xlabels at=edge top,
        xticklabels at=edge top,
        vertical sep=5pt,
    },
    axis x line*=top,
    xlabel near ticks,
    major x tick style=transparent,
    height=6cm,
    width=\linewidth,
    xmin=0, xmax=5,
    % enlarge y limits={value=0.2,upper},
    tick align=outside,
    tick style={white},
    ytick=\empty,
    xtick=\empty,
    xticklabels={},
    yticklabels={},
    % restrict y to domain*=0:2,
    clip=false,
]
\nextgroupplot[
    ylabel={}, % XXX: update 1
    ylabel shift={6mm},
    ymin=0, ymax=1,
]
\end{groupplot}

\begin{groupplot}[
    group style={
        group name=plots,
        group size=1 by 1,
        xlabels at=edge bottom,
        xticklabels at=edge bottom,
        vertical sep=5pt
    },
    axis x line*=bottom,
    xlabel near ticks,
    major x tick style=transparent,
    height=6cm,
    width=\linewidth,
    %xmin=0, %xmax=2,
    xlabel={\footnotesize File Size (bytes)},
    symbolic x coords={4K,512K,5M,40M},
    xticklabels={\footnotesize{4K}, \footnotesize{512K}, \footnotesize{5M}, \footnotesize{40M}},
    xtick=data,
    enlarge x limits=0.2, % add some breathing room along the x axis's sides
    extra y ticks={1},
    extra y tick style={grid=major, grid style={dashed, black}},
    extra y tick label={ 1 },
    tick align=outside,
    tick style={ black },
    legend cell align=center,
    legend style={ column sep=1ex },
    ymajorgrids=true,
    grid style={ dotted, gray },
    every node near coord/.append style={font=\tiny},
    % magic to make the numbers appear above the overly long bars:
    visualization depends on={rawy \as \rawy}, % save original y values
    restrict y to domain*={ % now clip/restrict any y value to ymax
        \pgfkeysvalueof{/pgfplots/ymin}:\ymaxbreak
    },
    after end axis/.code={ % draw squiggly line indicating break
        \draw [semithick, white, decoration={snake,amplitude=0.1mm,segment length=0.75mm,post length=0.375mm}, decorate] (rel axis cs:0,1.01) -- (rel axis cs:1,1.01);
    },
    nodes near coords={\color{.!75!black}\pgfmathprintnumber\rawy}, % print the original y values (darkened in case they're too light)...
    nodes near coords greater equal only=\ymax, % ... but ONLY if they're >= ymax
    clip=false, % allow clip to protrude beyond ymax
    %
    % Custom stuff to edit per template
    % XXX: update 2
    ymin=0, ymax=\ymax,
    ytick={ 0, 0.5, 1.5, 2 },
    yticklabels={ 0, 0.5, 1.5, 2 },
]
\nextgroupplot[
    % XXX: update 3
    ybar=1pt, % change space between bars
    bar width=2.25pt,
    legend entries={
        {\scriptsize NILFS/writes},
        {\scriptsize F2FS/writes},
        {\scriptsize Ext4OJ/writes},
        {\scriptsize Ext4FJ/writes},
    },
    legend style={
        draw=none,
        legend columns=2,
        at={(0.5,1.35)},
        anchor=north
    },
    clip=false
]

% XXX: update 4
\addplot[fill=orangeDark, every node near coord/.append style={color=orangeDark}]
table[x=size, y=NILFS, col sep=space] {data/sb/microbench-gamut-sequential-w.dat};
\addplot[fill=purpleDark, every node near coord/.append style={color=purpleDark}]
table[x=size, y=F2FS, col sep=space] {data/sb/microbench-gamut-sequential-w.dat};
\addplot[draw=orangeLight, pattern=crosshatch, pattern color=orangeLight, every node near coord/.append style={color=orangeLight, yshift=0.15cm}]
table[x=size, y=Ext4OJ, col sep=space] {data/sb/microbench-gamut-sequential-w.dat};
\addplot[draw=purpleLight, pattern=crosshatch, pattern color=purpleDark, every node near coord/.append style={color=purpleDark}]
table[x=size, y=Ext4FJ, col sep=space] {data/sb/microbench-gamut-sequential-w.dat};

\end{groupplot}%
\end{tikzpicture}%
}
        \caption{Sequential writes.}
       \label{fig:microbench-gamut-sequential-w}
    \end{subfigure}\\[1ex]
    \hspace*{-0.9em}%
    \begin{subfigure}{0.5\linewidth}
        \vspace{0.5em}
        \centering
        {\begin{tikzpicture}[baseline]

% XXX: update 0
\pgfmathsetmacro{\ymax}{2} % set the maximum y value
\pgfmathsetmacro{\ymaxbreak}{2.05} % set the y value at which overflow is drawn

\begin{groupplot}[
    group style={
        group name=plots,
        group size=1 by 1,
        xlabels at=edge top,
        xticklabels at=edge top,
        vertical sep=5pt,
    },
    axis x line*=top,
    xlabel near ticks,
    major x tick style=transparent,
    height=6cm,
    width=\linewidth,
    xmin=0, xmax=5,
    % enlarge y limits={value=0.2,upper},
    tick align=outside,
    tick style={white},
    ytick=\empty,
    xtick=\empty,
    xticklabels={},
    yticklabels={},
    % restrict y to domain*=0:2,
    clip=false,
]
\nextgroupplot[
    ylabel={\footnotesize Latency (normalized to dm-crypt)}, % XXX: update 1
    ylabel shift={6mm},
    ymin=0, ymax=1,
]
\end{groupplot}

\begin{groupplot}[
    group style={
        group name=plots,
        group size=1 by 1,
        xlabels at=edge bottom,
        xticklabels at=edge bottom,
        vertical sep=5pt
    },
    axis x line*=bottom,
    xlabel near ticks,
    major x tick style=transparent,
    height=6cm,
    width=\linewidth,
    %xmin=0, %xmax=2,
    xlabel={\footnotesize File Size (bytes)},
    symbolic x coords={4K,512K,5M,40M},
    xticklabels={\footnotesize{4K}, \footnotesize{512K}, \footnotesize{5M}, \footnotesize{40M}},
    xtick=data,
    enlarge x limits=0.2, % add some breathing room along the x axis's sides
    extra y ticks={1},
    extra y tick style={grid=major, grid style={dashed, black}},
    extra y tick label={ 1 },
    tick align=outside,
    tick style={ black },
    legend cell align=center,
    legend style={ column sep=1ex },
    ymajorgrids=true,
    grid style={ dotted, gray },
    every node near coord/.append style={font=\tiny},
    % magic to make the numbers appear above the overly long bars:
    visualization depends on={rawy \as \rawy}, % save original y values
    restrict y to domain*={ % now clip/restrict any y value to ymax
        \pgfkeysvalueof{/pgfplots/ymin}:\ymaxbreak
    },
    after end axis/.code={ % draw squiggly line indicating break
        \draw [semithick, white, decoration={snake,amplitude=0.1mm,segment length=0.75mm,post length=0.375mm}, decorate] (rel axis cs:0,1.01) -- (rel axis cs:1,1.01);
    },
    nodes near coords={\color{.!75!black}\pgfmathprintnumber\rawy}, % print the original y values (darkened in case they're too light)...
    nodes near coords greater equal only=\ymax, % ... but ONLY if they're >= ymax
    clip=false, % allow clip to protrude beyond ymax
    %
    % Custom stuff to edit per template
    % XXX: update 2
    ymin=0, ymax=\ymax,
    ytick={ 0, 0.5, 1.5, 2 },
    yticklabels={ 0, 0.5, 1.5, 2 },
]
\nextgroupplot[
    % XXX: update 3
    ybar=1pt, % change space between bars
    bar width=2.25pt,
    clip=false
]

% XXX: update 4
\addplot[fill=orangeDark, every node near coord/.append style={color=orangeDark}]
table[x=size, y=NILFS, col sep=space] {img/microbench-gamut-random-r.dat};
\addplot[fill=purpleDark, every node near coord/.append style={color=purpleDark}]
table[x=size, y=F2FS, col sep=space] {img/microbench-gamut-random-r.dat};
\addplot[draw=orangeLight, pattern=crosshatch, pattern color=orangeLight, every node near coord/.append style={color=orangeLight}]
table[x=size, y=Ext4OJ, col sep=space] {img/microbench-gamut-random-r.dat};
\addplot[draw=purpleLight, pattern=crosshatch, pattern color=purpleDark, every node near coord/.append style={color=purpleDark}]
table[x=size, y=Ext4FJ, col sep=space] {img/microbench-gamut-random-r.dat};

\end{groupplot}%
\end{tikzpicture}%
}
        \caption{Random reads.}
       \label{fig:microbench-gamut-random-r}
    \end{subfigure}%
    \begin{subfigure}{0.5\linewidth}
        \centering
        {\begin{tikzpicture}[baseline]

% XXX: update 0
\pgfmathsetmacro{\ymax}{2} % set the maximum y value
\pgfmathsetmacro{\ymaxbreak}{2.04} % set the y value at which overflow is drawn

\begin{groupplot}[
    group style={
        group name=plots,
        group size=1 by 1,
        xlabels at=edge top,
        xticklabels at=edge top,
        vertical sep=5pt,
    },
    axis x line*=top,
    xlabel near ticks,
    major x tick style=transparent,
    height=6cm,
    width=\linewidth,
    xmin=0, xmax=5,
    % enlarge y limits={value=0.2,upper},
    tick align=outside,
    tick style={white},
    ytick=\empty,
    xtick=\empty,
    xticklabels={},
    yticklabels={},
    % restrict y to domain*=0:2,
    clip=false,
]
\nextgroupplot[
    ylabel={}, % XXX: update 1
    ylabel shift={6mm},
    ymin=0, ymax=1,
]
\end{groupplot}

\begin{groupplot}[
    group style={
        group name=plots,
        group size=1 by 1,
        xlabels at=edge bottom,
        xticklabels at=edge bottom,
        vertical sep=5pt
    },
    axis x line*=bottom,
    xlabel near ticks,
    major x tick style=transparent,
    height=6cm,
    width=\linewidth,
    %xmin=0, %xmax=2,
    xlabel={\footnotesize File Size (bytes)},
    symbolic x coords={4K,512K,5M,40M},
    xticklabels={\footnotesize{4K}, \footnotesize{512K}, \footnotesize{5M}, \footnotesize{40M}},
    xtick=data,
    enlarge x limits=0.2, % add some breathing room along the x axis's sides
    extra y ticks={1},
    extra y tick style={grid=major, grid style={dashed, black}},
    extra y tick label={ 1 },
    tick align=outside,
    tick style={ black },
    legend cell align=center,
    legend style={ column sep=1ex },
    ymajorgrids=true,
    grid style={ dotted, gray },
    every node near coord/.append style={font=\tiny},
    % magic to make the numbers appear above the overly long bars:
    visualization depends on={rawy \as \rawy}, % save original y values
    restrict y to domain*={ % now clip/restrict any y value to ymax
        \pgfkeysvalueof{/pgfplots/ymin}:\ymaxbreak
    },
    after end axis/.code={ % draw squiggly line indicating break
        \draw [semithick, white, decoration={snake,amplitude=0.1mm,segment length=0.75mm,post length=0.375mm}, decorate] (rel axis cs:0,1.01) -- (rel axis cs:1,1.01);
    },
    nodes near coords={\color{.!75!black}\pgfmathprintnumber\rawy}, % print the original y values (darkened in case they're too light)...
    nodes near coords greater equal only=\ymax, % ... but ONLY if they're >= ymax
    clip=false, % allow clip to protrude beyond ymax
    %
    % Custom stuff to edit per template
    % XXX: update 2
    ymin=0, ymax=\ymax,
    ytick={ 0, 0.5, 1.5, 2 },
    yticklabels={ 0, 0.5, 1.5, 2 },
]
\nextgroupplot[
    % XXX: update 3
    ybar=1pt, % change space between bars
    bar width=2.25pt,
    clip=false
]

% XXX: update 4
\addplot[fill=orangeDark, every node near coord/.append style={color=orangeDark}]
table[x=size, y=NILFS, col sep=space] {img/microbench-gamut-random-w.dat};
\addplot[fill=purpleDark, every node near coord/.append style={color=purpleDark}]
table[x=size, y=F2FS, col sep=space] {img/microbench-gamut-random-w.dat};
\addplot[draw=orangeLight, pattern=crosshatch, pattern color=orangeLight, every node near coord/.append style={color=orangeLight, yshift=0.15cm}]
table[x=size, y=Ext4OJ, col sep=space] {img/microbench-gamut-random-w.dat};
\addplot[draw=purpleLight, pattern=crosshatch, pattern color=purpleDark, every node near coord/.append style={color=purpleDark}]
table[x=size, y=Ext4FJ, col sep=space] {img/microbench-gamut-random-w.dat};

\end{groupplot}%
\end{tikzpicture}%
}
        \caption{Random writes.}
       \label{fig:microbench-gamut-random-w}
    \end{subfigure}
    \caption{Comparison of four filesystems running on top of
      StrongBox performance is normalized to the same file system
      running on dm-crypt. Points below the line signify StrongBox
      outperforming dm-crypt. Points above the line signify dm-crypt
      outperforming StrongBox.}
   \label{fig:microbench-gamut}
\end{figure}

\subsection{StrongBox Write Performance}

\figref{microbench-f2fs} shows the performance of StrongBox in comparison to
dm-crypt under the modern F2FS LFS broken down by workload file size. Similar to
read performance under the F2FS, we see StrongBox improves on the performance of
dm-crypt's AES-XTS implementation across sequential and random write operations
on all file sizes. Hence, StrongBox under F2FS is holistically faster than
dm-crypt under F2FS\@. Specifically, $1.55\times$ (1.80h/1.16h) for sequential
5GB, $1.33\times$ (3.19s/2.39s) for sequential 40MB, $1.21\times$
(412.51ms/341.56ms) for sequential 5MB, $1.15\times$ (65.23ms/56.63ms) for
sequential 512KB, and $1.19\times$ (30.30ms/25.46ms) for sequential 4KB\@.

\figref{microbench-gamut} provides an expanded performance profile for
StrongBox, testing a gamut of filesystems broken down by workload file size.
Unlike read performance, write performance under certain filesystems is more of
a mixed bag. For 40MB sequential writes, StrongBox outperforms dm-crypt's
AES-XTS implementation by $1.33\times$ (3.19s/2.39s) for F2FS and $1.18\times$
(4.39s/3.74s) for NILFS\@. When it comes to Ext4, StrongBox's write performance
drops precipitously with a $3.6\times$ \textit{slowdown} for both ordered
journaling and full journaling modes (respectively: 12.64s/3.51s, 24.89s/6.88s).
For non-LFS 4KB writes, the performance degradation is even more pronounced with
a $8.09\times$ (118.48ms/14.65ms) slowdown for ordered journaling and
$14.5\times$ (143.15ms/9.87ms) slowdown for full journaling.

This slowdown occurs in Ext4 because, while writes in StrongBox from non-LFS
filesystems have a metadata overhead that is comparable to that of forward
writes in an LFS filesystem, Ext4 is not an append-only or append-mostly
filesystem. This means that, at any time, Ext4 will initiate one or more
overwrites anywhere on the drive (see \tblref{overwrites}). As described in
\secref{sb-design}, overwrites once detected trigger the rekeying process, which is
a relatively expensive operation. Multiple overwrites compound this expense
further. This makes Ext4 and other filesystems that do not exhibit at least
append-mostly behavior unsuitable for use with StrongBox. We include it in our
result set regardless to illustrate the drastic performance impact of frequent
overwrites on StrongBox.

For both sequential and random 4KB writes among the LFSes, the performance
improvement over dm-crypt's AES-XTS implementation for LFSes deflates. For the
more modern F2FS atop StrongBox, there is a $1.19\times$ (30.30ms/25.46ms)
improvement. For the older NILFS filesystem atop StrongBox, there is a
$2.38\times$ (27.19ms/11.44ms) slowdown. This is where we begin to see the
overhead associated with tracking writes and detecting overwrites potentially
becoming problematic, though the overhead is negligible depending on choice of
LFS and workload characteristics.

These results show that StrongBox is sensitive to the behavior of the LFS that
is mounted atop it, and that any practical use of StrongBox would require an
extra profiling step to determine which LFS works best with a specific workload.
With the correct selection of LFS, such as F2FS for workloads dominated by small
write operations, potential slowdowns when compared to mounting that same
filesystem over dm-crypt's AES-XTS can be effectively mitigated.

\subsection{On Replacing dm-crypt and Ext4}

\begin{figure}[ht]
    \textbf{StrongBox F2FS vs dm-crypt AES-XTS Ext4-OJ}\par\medskip
    \begin{subfigure}{\linewidth}
        \centering
        {\begin{tikzpicture}[baseline]

% XXX: update 0
\pgfmathsetmacro{\ymax}{3.5} % set the maximum y value
\pgfmathsetmacro{\ymaxbreak}{3.575} % set the y value at which overflow is drawn

\begin{groupplot}[
    group style={
        group name=plots,
        group size=1 by 1,
        xlabels at=edge top,
        xticklabels at=edge top,
        vertical sep=5pt
    },
    axis x line*=top,
    xlabel near ticks,
    major x tick style=transparent,
    height=6cm,
    width=\linewidth,
    xmin=0, xmax=5,
    % enlarge y limits={value=0.2,upper},
    tick align=outside,
    tick style={white},
    ytick=\empty,
    xtick=\empty,
    xticklabels={},
    yticklabels={},
    % restrict y to domain*=0:2,
    clip=false,
]
\nextgroupplot[
    ylabel={\footnotesize Latency (normalized to Ext4)}, % XXX: update 1
    ylabel shift={6mm},
    ymin=0, ymax=1,
]
\end{groupplot}

\begin{groupplot}[
    group style={
        group name=plots,
        group size=1 by 1,
        xlabels at=edge bottom,
        xticklabels at=edge bottom,
        vertical sep=5pt
    },
    axis x line*=bottom,
    xlabel near ticks,
    major x tick style=transparent,
    height=6cm,
    width=\linewidth,
    %xmin=0, %xmax=2,
    xlabel={\footnotesize File Size (bytes)},
    symbolic x coords={4K,512K,5M,40M,Mean},
    xtick=data,
    enlarge x limits=0.125, % add some breathing room along the x axis's sides
    extra y ticks={1},
    extra y tick style={grid=major, grid style={dashed, black}},
    extra y tick label={ 1 },
    tick align=outside,
    tick style={ black },
    legend cell align=center,
    legend style={ column sep=1ex },
    ymajorgrids=true,
    grid style={ dotted, gray },
    every node near coord/.append style={font=\tiny},
    % magic to make the numbers appear above the overly long bars:
    visualization depends on={rawy \as \rawy}, % save original y values
    restrict y to domain*={ % now clip/restrict any y value to ymax
        \pgfkeysvalueof{/pgfplots/ymin}:\ymaxbreak
    },
    after end axis/.code={ % draw squiggly line indicating break
        \draw [semithick, white, decoration={snake,amplitude=0.1mm,segment length=0.75mm,post length=0.375mm}, decorate] (rel axis cs:0,1.01) -- (rel axis cs:1,1.01);
    },
    nodes near coords={\color{.!75!black}\pgfmathprintnumber\rawy}, % print the original y values (darkened in case they're too light)...
    nodes near coords greater equal only=\ymax, % ... but ONLY if they're >= ymax
    clip=false, % allow clip to protrude beyond ymax
    %
    % Custom stuff to edit per template
    % XXX: update 2
    ymin=0.5, ymax=\ymax,
    ytick={ 0.5, 1.5, 2, 2.5, 3.0, 3.5 },
    yticklabels={ 0.5, 1.5, 2, 2.5, 3.0, 3.5 },
]
\nextgroupplot[
    % XXX: update 3
    ybar=1pt, % change space between bars
    bar width=4.5pt, % change size of bars
    legend entries={
        {\scriptsize unencrypted F2FS/reads},
        {\scriptsize StrongBox F2FS/reads},
        {\scriptsize dm-crypt Ext4/reads},
        {\scriptsize unencrypted F2FS/writes},
        {\scriptsize StrongBox F2FS/writes},
        {\scriptsize dm-crypt Ext4/writes},
    },
    legend style={
        draw=none,
        legend columns=3,
        at={(0.46,1.3)},
        anchor=north
    },
    clip=false
]

% XXX: update 4
\addplot[fill=orangeLight, every node near coord/.append style={color=orangeLight}]
table[x=size, y=uF2FSr, col sep=space] {data/sb/microbench-ext4-sequential.dat};
\addplot[fill=purpleDark, every node near coord/.append style={color=purpleDark}]
table[x=size, y=uF2FSw, col sep=space] {data/sb/microbench-ext4-sequential.dat};
\addplot[fill=purpleLight, every node near coord/.append style={color=purpleLight}]
table[x=size, y=sF2FSr, col sep=space] {data/sb/microbench-ext4-sequential.dat};
\addplot[draw=orangeLight, pattern=crosshatch, pattern color=orangeLight, every node near coord/.append style={color=orangeLight}]
table[x=size, y=sF2FSw, col sep=space] {data/sb/microbench-ext4-sequential.dat};
\addplot[draw=purpleDark, pattern=crosshatch, pattern color=purpleDark, every node near coord/.append style={color=purpleDark}]
table[x=size, y=dExt4r, col sep=space] {data/sb/microbench-ext4-sequential.dat};
\addplot[draw=purpleLight, pattern=crosshatch, pattern color=purpleLight, every node near coord/.append style={color=purpleLight}]
table[x=size, y=dExt4w, col sep=space] {data/sb/microbench-ext4-sequential.dat};

\end{groupplot}%
\end{tikzpicture}%
}
        \caption{Sequential I/O F2FS vs Ext4 result set.}
       \label{fig:microbench-ext4-sequential}
    \end{subfigure}\\[1ex]
    \begin{subfigure}{\linewidth}
        \centering
        {\begin{tikzpicture}[baseline]

% XXX: update 0
\pgfmathsetmacro{\ymax}{3.5} % set the maximum y value
\pgfmathsetmacro{\ymaxbreak}{3.575} % set the y value at which overflow is drawn

\begin{groupplot}[
    group style={
        group name=plots,
        group size=1 by 1,
        xlabels at=edge top,
        xticklabels at=edge top,
        vertical sep=5pt
    },
    axis x line*=top,
    xlabel near ticks,
    major x tick style=transparent,
    height=6cm,
    width=\linewidth,
    xmin=0, xmax=5,
    % enlarge y limits={value=0.2,upper},
    tick align=outside,
    tick style={white},
    ytick=\empty,
    xtick=\empty,
    xticklabels={},
    yticklabels={},
    % restrict y to domain*=0:2,
    clip=false,
]
\nextgroupplot[
    ylabel={\footnotesize Latency (normalized to Ext4)}, % XXX: update 1
    ylabel shift={6mm},
    ymin=0, ymax=1,
]
\end{groupplot}

\begin{groupplot}[
    group style={
        group name=plots,
        group size=1 by 1,
        xlabels at=edge bottom,
        xticklabels at=edge bottom,
        vertical sep=5pt
    },
    axis x line*=bottom,
    xlabel near ticks,
    major x tick style=transparent,
    height=6cm,
    width=\linewidth,
    %xmin=0, %xmax=2,
    xlabel={\footnotesize File Size (bytes)},
    symbolic x coords={4K,512K,5M,40M,Mean},
    xtick=data,
    enlarge x limits=0.125, % add some breathing room along the x axis's sides
    extra y ticks={1},
    extra y tick style={grid=major, grid style={dashed, black}},
    extra y tick label={ 1 },
    tick align=outside,
    tick style={ black },
    legend cell align=center,
    legend style={ column sep=1ex },
    ymajorgrids=true,
    grid style={ dotted, gray },
    every node near coord/.append style={font=\tiny},
    % magic to make the numbers appear above the overly long bars:
    visualization depends on={rawy \as \rawy}, % save original y values
    restrict y to domain*={ % now clip/restrict any y value to ymax
        \pgfkeysvalueof{/pgfplots/ymin}:\ymaxbreak
    },
    after end axis/.code={ % draw squiggly line indicating break
        \draw [semithick, white, decoration={snake,amplitude=0.1mm,segment length=0.75mm,post length=0.375mm}, decorate] (rel axis cs:0,1.01) -- (rel axis cs:1,1.01);
    },
    nodes near coords={\color{.!75!black}\pgfmathprintnumber\rawy}, % print the original y values (darkened in case they're too light)...
    nodes near coords greater equal only=\ymax, % ... but ONLY if they're >= ymax
    clip=false, % allow clip to protrude beyond ymax
    %
    % Custom stuff to edit per template
    % XXX: update 2
    ymin=0.5, ymax=\ymax,
    ytick={ 0.5, 1.5, 2, 2.5, 3.0, 3.5 },
    yticklabels={ 0.5, 1.5, 2, 2.5, 3.0, 3.5 },
]
\nextgroupplot[
    % XXX: update 3
    ybar=1pt, % change space between bars
    bar width=4.5pt, % change size of bars
    clip=false
]

% XXX: update 4
\addplot[fill=orangeLight, every node near coord/.append style={color=orangeLight}]
table[x=size, y=uF2FSr, col sep=space] {data/sb/microbench-ext4-random.dat};
\addplot[fill=purpleDark, every node near coord/.append style={color=purpleDark}]
table[x=size, y=uF2FSw, col sep=space] {data/sb/microbench-ext4-random.dat};
\addplot[fill=purpleLight, every node near coord/.append style={color=purpleLight}]
table[x=size, y=sF2FSr, col sep=space] {data/sb/microbench-ext4-random.dat};
\addplot[draw=orangeLight, pattern=crosshatch, pattern color=orangeLight, every node near coord/.append style={color=orangeLight}]
table[x=size, y=sF2FSw, col sep=space] {data/sb/microbench-ext4-random.dat};
\addplot[draw=purpleDark, pattern=crosshatch, pattern color=purpleDark, every node near coord/.append style={color=purpleDark}]
table[x=size, y=dExt4r, col sep=space] {data/sb/microbench-ext4-random.dat};
\addplot[draw=purpleLight, pattern=crosshatch, pattern color=purpleLight, every node near coord/.append style={color=purpleLight}]
table[x=size, y=dExt4w, col sep=space] {data/sb/microbench-ext4-random.dat};

\end{groupplot}%
\end{tikzpicture}%
}
        \caption{Random I/O F2FS vs Ext4 result set.}
       \label{fig:microbench-ext4-random}
    \end{subfigure}
    \caption{Comparison of Ext4 on dm-crypt and F2FS on StrongBox.
      Results are normalized to unencrypted Ext4 performance. Unecrypted F2FS
      results are shown for reference. Points below the line are outperforming
      unencrypted Ext4. Points above the line are underperforming compared
      to unencrypted Ext4.}
   \label{fig:microbench-ext4}
\end{figure}

\figref{microbench-ext4} describes the performance benefit of using StrongBox
with F2FS over the popular dm-crypt with Ext4 in ordered journaling mode
combination for both sequential and random read and write operations of various
sizes. Other than 4KB and 512KB write operations, which are instances where
baseline F2FS without StrongBox is simply slower than baseline Ext4 without
dm-crypt, StrongBox with F2FS outperforms dm-crypt's AES-XTS implementation with
Ext4.

These results show that configurations taking advantage of the popular
combination of dm-crypt, AES-XTS, and Ext4 could see a significant improvement
in read performance without a degradation in write performance except in cases
where small ($\leq512KB$) writes dominate the workload.

Note, however, that several implicit assumptions exist in our design. For one,
we presume there is ample memory at hand to house the Merkle Tree and all other
data abstractions used by StrongBox. Efficient memory use was not a goal of our
implementation of StrongBox. In an implementation aiming to be production ready,
much more memory efficient data structures would be utilized.

It is also for this reason that populating the Merkle Tree necessitates a rather
lengthy mounting process. In our tests, a 1GB backing store on the odroid system
can take as long as 15 seconds to mount.

\subsection{Performance in StrongBox: ChaCha20 vs AES}
\begin{figure}[t]
    \textbf{ChaCha20 vs AES: StrongBox F2FS Sequential Test}\par\medskip
    \centering
    {\begin{tikzpicture}[baseline]

\pgfmathsetmacro{\ymax}{2} % set the maximum y value
\pgfmathsetmacro{\ymaxbreak}{2.1} % set the y value at which overflow is drawn

\begin{groupplot}[
    group style={
        group name=plots,
        group size=1 by 1,
        xlabels at=edge top,
        xticklabels at=edge top,
        vertical sep=5pt,
    },
    axis x line*=top,
    xlabel near ticks,
    major x tick style=transparent,
    height=6cm,
    width=\linewidth,
    xmin=0, xmax=5,
    % enlarge y limits={value=0.2,upper},
    tick align=outside,
    tick style={white},
    ytick=\empty,
    xtick=\empty,
    xticklabels={},
    yticklabels={},
    % restrict y to domain*=0:2,
    clip=false,
]
\nextgroupplot[
    ylabel={\footnotesize Latency (normalized to ChaCha20)},
    ylabel shift={6mm},
    ymin=0, ymax=1,
]
\end{groupplot}

\begin{groupplot}[
    group style={
        group name=plots,
        group size=1 by 1,
        xlabels at=edge bottom,
        xticklabels at=edge bottom,
        vertical sep=5pt
    },
    axis x line*=bottom,
    xlabel near ticks,
    major x tick style=transparent,
    height=6cm,
    width=\linewidth,
    %xmin=0, %xmax=2,
    tick align=outside,
    tick style={ black },
    legend cell align=center,
    legend style={ column sep=1ex },
    ymajorgrids=true,
    grid style={ dotted, gray },
    every node near coord/.append style={font=\tiny},
    % magic to make the numbers appear above the overly long bars:
    visualization depends on={rawy \as \rawy}, % save original y values
    restrict y to domain*={ % now clip/restrict any y value to ymax
        \pgfkeysvalueof{/pgfplots/ymin}:\ymaxbreak
    },
    after end axis/.code={ % draw squiggly line indicating break
        \draw [semithick, white, decoration={snake,amplitude=0.1mm,segment length=0.75mm,post length=0.375mm}, decorate] (rel axis cs:0,1.01) -- (rel axis cs:1,1.01);
    },
    nodes near coords={\color{.!75!black}\pgfmathprintnumber\rawy}, % print the original y values (darkened in case they're too light)...
    nodes near coords greater equal only=\ymax, % ... but ONLY if they're >= ymax
    clip=false, % allow clip to protrude beyond ymax
    %
    % Custom stuff to edit per template
    xlabel={\footnotesize File Size (bytes)},
    symbolic x coords={4K,512K,5M,40M,Mean},
    xtick=data,
    enlarge x limits=0.2, % add some breathing room along the x axis's sides
    ymin=0, ymax=\ymax,
    ytick={ 0, 0.5, 1.5, 2 },
    yticklabels={ 0, 0.5, 1.5, 2 },
    extra y ticks={1},
    extra y tick style={grid=major, grid style={dashed, black}},
    extra y tick label={ 1 },
]
\nextgroupplot[
    ybar=1pt, % change space between bars
    bar width=4.5pt, % change size of bars
    legend entries={
        {\scriptsize AES-XTS/reads},
        {\scriptsize AES-CTR/reads},
        {\scriptsize AES-XTS/writes},
        {\scriptsize AES-CTR/writes},
    },
    legend style={
        draw=none,
        legend columns=2,
        at={(0.525,1.3)},
        anchor=north
    },
]

\addplot[fill=orangeDark, every node near coord/.append style={color=orangeDark}]
table[x=size, y=XTSr, col sep=space] {img/microbench-aes-vs-chacha.dat};
\addplot[fill=purpleDark, every node near coord/.append style={color=purpleDark}]
table[x=size, y=CTRr, col sep=space] {img/microbench-aes-vs-chacha.dat};
\addplot[draw=orangeDark, pattern=crosshatch, pattern color=orangeDark, every node near coord/.append style={color=orangeDark}]
table[x=size, y=XTSw, col sep=space] {img/microbench-aes-vs-chacha.dat};
\addplot[draw=purpleDark, pattern=crosshatch, pattern color=purpleDark, every node near coord/.append style={color=purpleDark}]
table[x=size, y=CTRw, col sep=space] {img/microbench-aes-vs-chacha.dat};

\end{groupplot}%
\end{tikzpicture}%
}
    \caption{Comparison of AES in XTS and CTR modes versus ChaCha20 in
      StrongBox; median latency of different sized whole file
      sequential read and write operations normalized to ChaCha20
      (default cipher in StrongBox). Points below the line signify AES
      outperforming ChaCha20. Points above the line signify ChaCha20
      outperforming AES.}
   \label{fig:microbench-aes-vs-chacha}
\end{figure}
\figref{microbench-gamut} and \figref{microbench-f2fs} give strong evidence for
our general performance improvement over dm-crypt not being an artifact of
filesystem choice. Excluding Ext4 as a non-LFS filesystem under which to run
StrongBox, our tests show that StrongBox outperforms dm-crypt under an LFS
filesystem in the vast majority of outcomes.

We then test to see if our general performance improvement can be attributed to
the use of a stream cipher over a block cipher. dm-crypt implements AES in XTS
mode to provide full drive encryption functionality.
\figref{microbench-aes-vs-chacha} describes the relationship between ChaCha20,
the cipher of choice for our implementation of StrongBox, and the AES cipher.
Swapping out ChaCha20 for AES-CTR resulted in slowdowns of up to $1.33\times$
for reads and $1.15\times$ for writes across all configurations, as described in
\figref{microbench-aes-vs-chacha}.

Finally, we test to see if our general performance improvement can be attributed
to our implementation of StrongBox rather than our choice of stream cipher. We
test this by implementing AES in XTS mode on top of StrongBox using OpenSSL EVP
(see: \figref{microbench-aes-vs-chacha}). StrongBox using OpenSSL AES-XTS
experiences slowdowns of up to $1.33\times$ for reads and $1.6\times$ for writes
compared to StrongBox using ChaCha20 (sequential, 40MB). Interestingly, while
significantly less performant, this slowdown is not entirely egregious, and
suggests that perhaps there are parts of the dm-crypt code base that would
benefit from further optimization; however, it is possible that necessary
choices to harden StrongBox for a production environment could slow it down as
well.

Considering hardware support for dedicated AES instructions,
\figref{microbench-aes-vs-chacha} shows StrongBox with AES-CTR outperforms
AES-XTS. Therefore, StrongBox should still outperform dmcrypt where AES hardware
support is available.

\subsection{Overhead with a Full Drive}
During I/O operations under an appropriate choice of LFS, we have shown that
full drive encryption provided by StrongBox outperforms full drive encryption
provided by dm-crypt. However, this is not necessarily the case when the backing
store becomes full and the LFS is forced to cope with an inability to write
forward as efficiently.

In the case of the F2FS LFS, upon approaching capacity and being unable to
perform garbage collection effectively, it resorts to writing blocks out to
where ever it can find free space in the backing store~\cite{F2FS}. It does this
instead of trying to maintain an append-only guarantee. This method of executing
writes is similar to how a typical non-LFS filesystem operates. When this
happens, the F2FS aggressively causes overwrites in StrongBox, which has a
drastic impact on performance.

\figref{microbench-f2fs-full} shows the impact of these (sequential) overwrites.
Read operation performance remains faster on a full StrongBox backing store
compared to dm-crypt. This is not the case with writes. Compared to StrongBox
under non-full conditions, 40MB sequential writes were slowed by $2.8\times$ as
StrongBox approached maximum capacity.

\begin{figure}[ht]
    \textbf{Near-Full Drive F2FS Test}\par\medskip
    \centering
    {\begin{tikzpicture}[baseline]

% XXX: update 0
\pgfmathsetmacro{\ymax}{2} % set the maximum y value
\pgfmathsetmacro{\ymaxbreak}{2.05} % set the y value at which overflow is drawn

\begin{groupplot}[
    group style={
        group name=plots,
        group size=1 by 1,
        xlabels at=edge top,
        xticklabels at=edge top,
        vertical sep=5pt
    },
    axis x line*=top,
    xlabel near ticks,
    major x tick style=transparent,
    height=6cm,
    width=\linewidth,
    xmin=0, xmax=5,
    % enlarge y limits={value=0.2,upper},
    tick align=outside,
    tick style={white},
    ytick=\empty,
    xtick=\empty,
    xticklabels={},
    yticklabels={},
    % restrict y to domain*=0:2,
    clip=false,
]
\nextgroupplot[
    ylabel={\footnotesize Latency (normalized to dm-crypt)}, % XXX: update 1
    ylabel shift={6mm},
    ymin=0, ymax=1,
]
\end{groupplot}

\begin{groupplot}[
    group style={
        group name=plots,
        group size=1 by 1,
        xlabels at=edge bottom,
        xticklabels at=edge bottom,
        vertical sep=5pt
    },
    axis x line*=bottom,
    xlabel near ticks,
    major x tick style=transparent,
    height=6cm,
    width=\linewidth,
    %xmin=0, %xmax=2,
    xlabel={\footnotesize File Size (bytes)},
    symbolic x coords={4K,512K,5M,40M,Mean},
    xtick=data,
    enlarge x limits=0.2, % add some breathing room along the x axis's sides
    extra y ticks={1},
    extra y tick style={grid=major, grid style={dashed, black}},
    extra y tick label={ 1 },
    tick align=outside,
    tick style={ black },
    legend cell align=center,
    legend style={ column sep=1ex },
    ymajorgrids=true,
    grid style={ dotted, gray },
    every node near coord/.append style={font=\tiny},
    % magic to make the numbers appear above the overly long bars:
    visualization depends on={rawy \as \rawy}, % save original y values
    restrict y to domain*={ % now clip/restrict any y value to ymax
        \pgfkeysvalueof{/pgfplots/ymin}:\ymaxbreak
    },
    after end axis/.code={ % draw squiggly line indicating break
        \draw [semithick, white, decoration={snake,amplitude=0.1mm,segment length=0.75mm,post length=0.375mm}, decorate] (rel axis cs:0,1.01) -- (rel axis cs:1,1.01);
    },
    nodes near coords={\color{.!75!black}\pgfmathprintnumber\rawy}, % print the original y values (darkened in case they're too light)...
    nodes near coords greater equal only=\ymax, % ... but ONLY if they're >= ymax
    clip=false, % allow clip to protrude beyond ymax
    %
    % Custom stuff to edit per template
    % XXX: update 2
    ymin=0, ymax=\ymax,
    ytick={ 0, 0.5, 1.5, 2 },
    yticklabels={ 0, 0.5, 1.5, 2 },
]
\nextgroupplot[
    % XXX: update 3
    ybar=1pt, % change space between bars
    bar width=4.5pt, % change size of bars
    legend entries={
        {\scriptsize unencrypted/reads},
        {\scriptsize StrongBox/reads},
        {\scriptsize unencrypted/writes},
        {\scriptsize StrongBox/writes},
    },
    legend style={
        draw=none,
        legend columns=2,
        at={(0.525,1.3)},
        anchor=north
    },
    clip=false
]

% XXX: update 4
\addplot[fill=orangeDark, every node near coord/.append style={color=orangeDark}]
table[x=size, y=uR, col sep=space] {data/sb/microbench-f2fs-full.dat};
\addplot[fill=purpleDark, every node near coord/.append style={color=purpleDark}]
table[x=size, y=sR, col sep=space] {data/sb/microbench-f2fs-full.dat};
\addplot[draw=orangeDark, pattern=crosshatch, pattern color=orangeDark, every node near coord/.append style={color=orangeDark}]
table[x=size, y=uW, col sep=space] {data/sb/microbench-f2fs-full.dat};
\addplot[draw=purpleDark, pattern=crosshatch, pattern color=purpleDark, every node near coord/.append style={color=purpleDark}]
table[x=size, y=sW, col sep=space] {data/sb/microbench-f2fs-full.dat};

\end{groupplot}%
\end{tikzpicture}%

}
    \caption{Comparison of F2FS baseline, atop dm-crypt, and atop
      StrongBox. All configurations are initialized with a near-full
      backing store; median latency of different sized whole file read
      and write operations normalized to dm-crypt. Points below the
      line are outperforming dm-crypt. Points above the line are
      underperforming compared to dm-crypt.}
   \label{fig:microbench-f2fs-full}
\end{figure}

\subsection{Threat Analysis}

\tblref{sb-security} lists possible attacks and their results. It can be inferred
from these results and StrongBox's design that StrongBox addresses its threat
model and maintains confidentiality and integrity guarantees.

\begin{table}[t]
  \caption{Attacks on StrongBox and their results} \label{tbl:sb-security}
  \footnotesize
  \centering
  \begin{tabu} to \linewidth { | X[l] | X[c] | X[c] | }
    \hline
    \textbf{Attack} & \textbf{Result} & \textbf{Explanation} \\
    \hline\hline
    Nugget user data in backing store is mutated out-of-band online & StrongBox Immediately fails with exception on successive IO request & The MTRH is inconsistent \\
    \hline
    Header metadata in backing store is mutated out-of-band online, making the MTRH inconsistent & StrongBox Immediately fails with exception on successive IO request & The MTRH is inconsistent \\
    \hline
    Backing store is rolled back to a previously consistent state while online & StrongBox Immediately fails with exception on successive IO request & TPMGLOBALVER and RPMB secure counter out of sync \\
    \hline
    Backing store is rolled back to a previously consistent state while offline, RPMB secure counter wildly out of sync & StrongBox refuses to mount; allows for force mount with root access & TPMGLOBALVER and RPMB secure counter out of sync \\
    \hline
    MTRH made inconsistent by mutating backing store out-of-band while offline, RPMB secure counter in sync & StrongBox refuses to mount & TPMGLOBALVER and RPMB secure counter are in sync, yet illegal data manipulation occurred \\
    \hline\hline
  \end{tabu}
\end{table}
