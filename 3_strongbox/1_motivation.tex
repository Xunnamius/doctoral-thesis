\section{Motivation} \label{sec:sb-motivation}

Full drive encryption (FDE)\footnote{The common term is full \emph{disk}
encryption, but this work targets SSDs, so we use \emph{drive}.} is an essential
technique for protecting the privacy of data at rest. For mobile devices,
maintaining data privacy is especially important as these devices contain
sensitive personal and financial data yet are easily lost or stolen. The current
standard for securing data at rest is to use the AES cipher in XTS
mode~\cite{XTS, NISTXTS}. Unfortunately, employing AES-XTS increases read/write
latency by 3--5$\times$ compared to unencrypted storage.

It is well known that authenticated encryption using \emph{stream}
ciphers---such as ChaCha20~\cite{ChaCha20}---is faster than using AES (see
\figref{motivation}). Indeed, Google made the case for stream ciphers over AES,
switching HTTPS connections on Chrome for Android to use a stream cipher for
better performance~\cite{google-blog}. Stream ciphers are not used for FDE,
however, for two reasons: (1) confidentiality and (2) performance. First, when
applied naively to stored data, stream ciphers are trivially vulnerable to
attacks---including \emph{many-time pad and rollback attacks}---that reveal the
plaintext by overwriting a secure storage location with the same key. Second, it
has been assumed that adding the meta-data required to resist these attacks
would ruin the stream cipher's performance advantage. Thus, the conventional
wisdom is that FDE necessarily incurs the overhead of AES-XTS or a similar
primitive.

We argue that two technological shifts in mobile device hardware overturn this
conventional wisdom, enabling confidential, high-performance storage with stream
ciphers. First, these devices commonly employ solid-state storage with Flash
Translation Layers (FTL), which operate similarly to Log-structured File Systems
(LFS)~\cite{LFS,F2FS,NILFS}. Second, mobile devices now support trusted
hardware, such as Trusted Execution Environments (TEE)~\cite{TEE,TrustZone} and
secure storage areas~\cite{eMMC-standard}. FTLs and LFSes are used to limit
sector/cell overwrites, hence extending the life of the drive. Most writes
simply appended to a log, reducing the occurrence of overwrites and the chance
for attacks. The presence of secure hardware means that drive encryption modules
have access to persistent, monotonically increasing counters that can be used to
prevent rollback attacks when overwrites do occur.

Given these trends, we propose StrongBox, a new method for securing data at
rest. StrongBox is a drop-in replacement for AES-XTS-backed FDE such as
dm-crypt~\cite{dmcrypt}; \ie it requires no interface changes. The primary
challenge is that even with a FTL or LFS running above an SSD, filesystem blocks
will occasionally be overwritten; \eg by segment cleaning or \emph{garbage
collection}. StrongBox overcomes this challenge by using a fast stream cipher
for confidentiality and performance with integrity preserving Message
Authentication Codes~\cite{MAC} or ``MAC tags'' and a secure, persistent
hardware counter to ensure integrity and prevent attacks. \emph{StrongBox's main
contribution is a system design enabling the first confidential, high-
performance drive encryption scheme based on a stream cipher.}

We demonstrate StrongBox's effectiveness on a mobile ARM big.LITTLE system---a
Samsung Exynos Octa 5---running Ubuntu Trusty 14.04 LTS, kernel 3.10.58. We use
ChaCha20~\cite{ChaCha20} as our stream cipher, Poly1305~\cite{Poly1305} as our
MAC algorithm, and the eMMC Replay Protected Memory Block partition to store a
secure counter \cite{eMMC-standard}. As StrongBox requires no change to any
existing interfaces, we benchmark it on two of the most popular LFSes:
NILFS~\cite{NILFS} and F2FS~\cite{F2FS}. We compare the performance of these
LFSes on top of AES-XTS (via dm-crypt) and StrongBox. Additionally, we compare
the performance of AES-XTS encrypted Ext4 filesystems with StrongBox and F2FS.
Our results show:

\begin{itemize}

  \item \emph{Improved read performance:} StrongBox provides decreased read
latencies across all tested filesystems in the majority of benchmarks when
compared to dm-crypt; \eg under F2FS, StrongBox provides as much as a
$2.36\times$ ($1.72\times$ average) speedup over AES-XTS\@.

  \item \emph{Equivalent write performance:} despite having to maintain more
metadata than FDE schemes based on AES-XTS, StrongBox achieves near parity or
provides an improvement in observed write latencies in the majority of
benchmarks; \eg under F2FS, StrongBox provides an average $1.27\times$ speedup
over AES-XTS\@.

\end{itemize}

StrongBox achieves these performance gains while providing a stronger integrity
guarantee than AES-XTS\@. Whereas XTS mode only hopes to randomize plaintext when
the ciphertext is altered~\cite{XTS}, StrongBox provides the security of
standard authenticated encryption. In addition, StrongBox's implementation is
available open-source.\footnote{\StrongBoxURI}

\subsection{Performance Potential}

We demonstrate the potential performance win from switching to a stream cipher
by comparing AES-XTS to ChaCha20+ Poly1305. We use an Exynos Octa processor with
an ARM big.LITTLE architecture---the same processor used in the Samsung Galaxy
line of phones. We encrypt and then decrypt 250MB of randomly generated bits 3
times and take the median time for each of encryption and decryption.
\figref{motivation} shows the distinct advantage of the stream cipher over
AES---a consistent $2.7\times$ reduction in run time.

\begin{figure}[ht]
  \centering
  \input{data/sb/heuristics-time.tex}
\caption{AES-XTS and ChaCha20+Poly1305 Comparison.} \label{fig:motivation}
\end{figure}

\subsection{Append-mostly Filesystems}
Of course, stream ciphers are not designed to encrypt data at rest.
If we naively implement block device encryption with a stream cipher,
overwriting the same memory location with the same key would trivially
allow an attacker to recover the secret key. Thus we believe stream
ciphers are best suited for encrypting block devices backing
Log-structured File Systems (LFSes), as these filesystems are designed
to append data to the end of a log rather than overwrite data. In
practice, some overwrites occur; \eg in metadata, but they are small
in number during normal execution.

To demonstrate this fact, we write 800MB of random data directly to the backing
store using four different file systems: Ext4, LogFS, NILFS, and F2FS. We count
the number of total writes to the underlying block device and the number of
times data is overwritten for each file system.


\begin{table}[th]
%\begin{wraptable}{r}{4cm}
\caption{File System Overwrite Behavior} \label{tbl:overwrites}
\footnotesize
\centering
\begin{tabular}{lrr}
  \textbf{File System} & \textbf{Total Write Ops} & \textbf{Overwrites} \\
  \hline
  \hline
  ext4    &  16,756 & 10,787\\
  LogFS   &   4,244 &     32\\
  NILFS   &   4,199 &     24\\
  F2FS    &   2.107 &      2\\
  \hline
  \hline
\end{tabular}
%\vskip -.7em
%\end{wraptable}
\end{table}

\tblref{overwrites} shows the data for this experiment. Ext4 has the highest
number of writes, but many of those are small writes for book-keeping purposes.
Ext4 also has the largest number of overwrites. Almost 65\% of the writes are to
a previously written location in the backing store. In contrast, all three
Log-structured file systems have very few overwrites.

\subsection{Threat Model}

A stream cipher can be more than twice as fast as AES-XTS while providing the
same confidentiality guarantee. The problem is that a stream cipher is not
secure if the same key is used to overwrite the same storage location.
Fortunately, FTLs and LFSes rarely overwrite the same location.

We cannot, however, ignore the fact that overwrites do occur. While
\tblref{overwrites} shows overwrites are rare during normal operation, we know
they will occur when garbage collecting the LFS. Thus, we will need some
metadata to track writes and ensure that data is handled securely if overwrites
occur. Therefore, we recognize three key challenges to replacing AES with a
stream cipher for FDE:

\begin{itemize}

\item Tracking writes to the block device to ensure that the same location is
never overwritten with the same key.

\item Ensuring that the metadata that tracks writes is secure and is not
subject to leaks or rollback attacks.

\item Accomplishing the above efficiently so that we maintain the  performance
advantage of the stream cipher.

\end{itemize}

The key to StrongBox is using a secure, persistent counter supported in modern
mobile hardware; \eg for limiting password attempts. This counter can track
writes, and thus \emph{versions} of the encrypted data. If an attacker tried to
\emph{roll back} the file system to overwrite the same location with the same
key, our StrongBox detects that the local version number is out of sync with the
global version number stored in the secure counter. In that case, StrongBox
refuses to initialize and the attack fails. The use of the hardware-supported
secure counter significantly raises the bar when it comes to rollback attacks,
requiring a costly and non-discrete physical attack on the hardware itself to be
effective. The actual structure of the metadata required to track writes and
maintain integrity is significantly more complicated than simply implementing a
counter and is the subject of the next section.

An additional challenge is that of crash recovery. StrongBox relies on the
overlying filesystem to manage data recovery in the event of a crash that leaves
user data in an inconsistent state. StrongBox handles metadata recovery after a
crash by giving the root user the option to accept the current metadata state as
the new consistent state, \ie{``force mounting''}. This allows the root user to
mount the filesystem and access data after an unexpected shutdown. An attacker
might try to take advantage of this feature by modifying the backing store,
forcing an inconsistent state, and hoping the root user will ignore it and force
mount the system anyway. StrongBox defends against this attack by preventing
force mounts when metadata state is wildly inconsistent with the global version
counter. Otherwise, the root user is warned if they attempt a force mount. Thus,
attacking StrongBox by forcing a crash can only be successful if the attacker
also has root permission, in which case security is already compromised. Crash
recovery is also detailed in the next section.
