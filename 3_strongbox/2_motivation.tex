\section{Motivation}\label{sec:motivation}

We detail the main motivations for StrongBox: stream ciphers' speed
compared to AES-XTS and Log-structured File Systems' append-mostly
nature. We then describe the challenges of replacing AES with a stream
cipher.

\subsection{Performance Potential}

We demonstrate the potential performance win from switching to a stream cipher
by comparing AES-XTS to ChaCha20+ Poly1305. We use an Exynos Octa processor with
an ARM big.LITTLE architecture---the same processor used in the Samsung Galaxy
line of phones. We encrypt and then decrypt 250MB of randomly generated bits 3
times and take the median time for each of encryption and decryption.
\figref{motivation} shows the distinct advantage of the stream cipher over
AES---a consistent $2.7\times$ reduction in run time.

\begin{figure}[t]
  \begin{tikzpicture}

\begin{groupplot}[
    group style={
        group name=plots,
        group size=1 by 1,
        xlabels at=edge top,
        xticklabels at=edge top,
        vertical sep=5pt
    },
axis x line* = top,
xlabel near ticks,
major x tick style = transparent,
height=3.5cm,
%width=0.95\columnwidth,
width=4.2cm,
xmin=0,
xmax=3,
enlargelimits=false,
tick align = outside,
tick style={white},
ytick=\empty,
xtick=\empty,
xticklabels={},
yticklabels={},
]
\nextgroupplot[ylabel={\footnotesize Time (s)},
ylabel shift={6mm},
ymin=0,
ymax=1,
]


\end{groupplot}

\begin{groupplot}[
    group style={
        group name=plots,
        group size=1 by 1,
        xlabels at=edge bottom,
        xticklabels at=edge bottom,
        vertical sep=5pt
    },
axis x line* = bottom,
xlabel near ticks,
major x tick style = transparent,
height=3.5cm,
%width=0.95\columnwidth,
width=4.2cm,
xmin=0,
xmax=3,
enlargelimits=false,
tick align = outside,
tick style={white},
ytick=\empty,
xticklabel shift={-5pt},
%x tick label style={rotate=0, anchor=south},
%xlabel={\footnotesize $Platform$}
xtick={1,2},
xticklabels={{\scriptsize $\mathsf{Encrypt}$},
{\scriptsize $\mathsf{Decrypt}$}},
ymin=0,
ymax=50,
ytick={0,12.5,25,37.5,50},
yticklabels={0,,25,,50},
legend cell align=left, 
legend style={ column sep=1ex },
ymajorgrids,
grid style={dashed},
]
\nextgroupplot[ybar=\pgflinewidth,
bar width=5pt,
legend entries = {{\scriptsize $\mathsf{AES-XTS}$},
{\scriptsize $\mathsf{ChaCha+Poly1305}$}
},
legend style={draw=none,legend columns=2,at={(0.5,1.35)},anchor=north},
]
\addplot table[x index=0,y index=4, col sep=space] {img/heuristics2.txt};
\addplot table[x index=0,y index=5, col sep=space] {img/heuristics2.txt};


\end{groupplot}

\end{tikzpicture}

\caption{AES-XTS and ChaCha20+Poly1305 Comparison.}\label{fig:motivation}
\end{figure}

\subsection{Append-mostly Filesystems}
Of course, stream ciphers are not designed to encrypt data at rest.
If we naively implement block device encryption with a stream cipher,
overwriting the same memory location with the same key would trivially
allow an attacker to recover the secret key. Thus we believe stream
ciphers are best suited for encrypting block devices backing
Log-structured File Systems (LFSes), as these filesystems are designed
to append data to the end of a log rather than overwrite data. In
practice, some overwrites occur; \eg in metadata, but they are small
in number during normal execution.

To demonstrate this fact, we write 800MB of random data directly to the backing
store using four different file systems: Ext4, LogFS, NILFS, and F2FS. We count
the number of total writes to the underlying block device and the number of
times data is overwritten for each file system.


\begin{table}[th]
%\begin{wraptable}{r}{4cm}
\caption{File System Overwrite Behavior}\label{tbl:overwrites}
\footnotesize
\centering
\begin{tabular}{lrr}
  \textbf{File System} & \textbf{Total Write Ops} & \textbf{Overwrites} \\
  \hline
  \hline
  ext4    &  16,756 & 10,787\\
  LogFS   &   4,244 &     32\\
  NILFS   &   4,199 &     24\\
  F2FS    &   2.107 &      2\\
  \hline 
  \hline
\end{tabular}
%\vskip -.7em
%\end{wraptable}
\end{table}

\tblref{overwrites} shows the data for this experiment. Ext4 has the highest
number of writes, but many of those are small writes for book-keeping purposes.
Ext4 also has the largest number of overwrites. Almost 65\% of the writes are to
a previously written location in the backing store. In contrast, all three
Log-structured file systems have very few overwrites.

\subsection{Threat Model}

A stream cipher can be more than twice as fast as AES-XTS while providing the
same confidentiality guarantee. The problem is that a stream cipher is not
secure if the same key is used to overwrite the same storage location.
Fortunately, FTLs and LFSes rarely overwrite the same location.

We cannot, however, ignore the fact that overwrites do occur. While
\tblref{overwrites} shows overwrites are rare during normal operation, we know
they will occur when garbage collecting the LFS. Thus, we will need some
metadata to track writes and ensure that data is handled securely if overwrites
occur. Therefore, we recognize three key challenges to replacing AES with a
stream cipher for FDE:

\begin{itemize}

\item Tracking writes to the block device to ensure that the same location is
never overwritten with the same key.

\item Ensuring that the metadata that tracks writes is secure and is not
subject to leaks or rollback attacks.

\item Accomplishing the above efficiently so that we maintain the  performance
advantage of the stream cipher.

\end{itemize}

The key to StrongBox is using a secure, persistent counter supported in modern
mobile hardware; \eg for limiting password attempts. This counter can track
writes, and thus \emph{versions} of the encrypted data. If an attacker tried to
\emph{roll back} the file system to overwrite the same location with the same
key, our StrongBox detects that the local version number is out of sync with the
global version number stored in the secure counter. In that case, StrongBox
refuses to initialize and the attack fails. The use of the hardware-supported
secure counter significantly raises the bar when it comes to rollback attacks,
requiring a costly and non-discrete physical attack on the hardware itself to be
effective. The actual structure of the metadata required to track writes and
maintain integrity is significantly more complicated than simply implementing a
counter and is the subject of the next section.

An additional challenge is that of crash recovery. StrongBox relies on the
overlying filesystem to manage data recovery in the event of a crash that leaves
user data in an inconsistent state. StrongBox handles metadata recovery after a
crash by giving the root user the option to accept the current metadata state as
the new consistent state, \ie{``force mounting''}. This allows the root user to
mount the filesystem and access data after an unexpected shutdown. An attacker
might try to take advantage of this feature by modifying the backing store,
forcing an inconsistent state, and hoping the root user will ignore it and force
mount the system anyway. StrongBox defends against this attack by preventing
force mounts when metadata state is wildly inconsistent with the global version
counter. Otherwise, the root user is warned if they attempt a force mount. Thus,
attacking StrongBox by forcing a crash can only be successful if the attacker
also has root permission, in which case security is already compromised. Crash
recovery is also detailed in the next section.

