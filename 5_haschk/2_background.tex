\section{Background} \label{sec:background}

In this section, we discuss Supply Chain Attacks (SCAs), including four case
studies describing real-world resource integrity attacks that take advantage of
the ineffectiveness of checksums. These motivate the \SYSTEM{} approach.

\subsection{Resource Integrity SCAs}

Modern software requires a complex globally distributed supply chain and
development ecosystem for organizations and other providers to design, develop,
deploy, and maintain products efficiently~\cite{SCA}. Such a globally
distributed ecosystem necessitates integration with potentially many third party
entities, be they specialty driver manufacturers, external content distribution
network (CDN) providers, third party database management software, download
mirrors, etc.

Critically, reliance on third parties, while often cost-effective and
feature-rich, also increases the risk of a security compromise at some point in
the supply chain~\cite{SCA, Stickler}. These types of compromises are known as
Supply Chain Attacks (SCA). In the context of software development, SCAs are the
compromise of an entity's software source code via cyber attack, insider threat,
upstream asset/package management compromise, trusted hardware/vendor
compromise, or some other attack on one or more phases of the software
development life cycle or ``supply chain'' to infect an unsuspecting end
user~\cite{NIST-SCA}.

Every year, major SCAs become more frequent and their fallout more widely
felt~\cite{SCA, NIST-SCA}. Whether major or minor, SCAs are hard to detect, even
harder to prevent, and have the goal of infecting and exploiting victims by
violating the trust between consumer and reputable software vendor.

\begin{table*}[t]
    \centering
    \begin{tabular}{|*{10}{c|}}
      \hline\textbf{Concept}
          & \textbf{Design} & \textbf{Development} & \textbf{Integration} &
          \textbf{Deployment} & \textbf{Maintenance} &
          \textbf{Retirement}\\\hline
      X&X&X&X&\ding{51}&\ding{51}&\ding{51}\\\hline
    \end{tabular}
    \caption{The generic software engineering supply chain.}
     \label{tbl:attacks}
\end{table*}

\tblref{attacks} details the phases of a generic software development supply
chain. For the purposes of this research, we focus exclusively on SCAs targeting
the deployment, maintenance, and retirement phases.

\subsection{Motivation: Cases}

Here, we select four relatively recent real-world attacks we believe most
effectively articulate the threat posed by resource integrity SCAs and how
\SYSTEM{} might have been used to more effectively mitigate fallout. We examine
each attack, noting the critical points of failure in their checksum-based
resource security models. \\

\noindent\textbf{Case 1: PhpMyAdmin.} For an unspecified amount of time circa
2012, a compromised download mirror in SourceForge's official HTTPS-protected
CDN was distributing a malicious version of the popular database administration
software phpMyAdmin~\cite{SCA-PMA3}. The administrator of the mirror in question
confirmed the attack was due to a vulnerability not shared by SourceForge's
other mirrors~\cite{SCA-PMA2}.

Attackers mutated the software image, injecting files that would allow any
attacker aware of their existence to remotely execute arbitrary PHP code on the
victim's system~\cite{SCA-PMA1}. SourceForge estimates approximately 400 unique
users downloaded this corrupted version of phpMyAdmin before the mirror was
disconnected from their CDN, potentially allowing attackers access to the
private customer data of any number of organizations~\cite{SCA-PMA2}.

While the attackers were able to penetrate a mirror in SourceForge's CDN, the
official phpMyAdmin website was entirely unaffected; the authoritative checksums
listed on the site's download page were similarly unaffected~\cite{SCA-PMA2}.
Hence, a user who was sufficiently motivated, had sufficient technical knowledge
of checksums and how to calculate them, and was also privy to the location of
the correct checksum for the official phpMyAdmin image \emph{might} have noticed
the discrepancy between the two digests. Clearly, a non-trivial number of users
do not meet these criteria. This attack demonstrates the problem of \emph{user
apathy}.\\

\noindent\textbf{Case 2: Linux Mint.} In 2016, the Linux Mint team discovered an
intrusion into their official HTTPS-protected distribution
server~\cite{SCA-MINT1}. Attackers mutated download links originally pointing to
the Linux Mint 17.3 Cinnamon edition ISO, redirecting unsuspecting users to a
separate system hosting a custom Mint ISO compiled with the IRC-based Linux
backdoor malware \emph{Tsunami}~\cite{SCA-MINT2}. The attack affected hundreds
of the downloads during that day, with the attackers claiming that a ``few
hundred'' Linux Mint installs were explicitly under their control. The primary
motivation behind the intrusion was the construction of a
botnet~\cite{SCA-MINT3}. The authoritative checksum displayed on the official
website was also mutated to corroborate the backdoored ISO~\cite{SCA-MINT3},
illustrating the \emph{co-hosting} problem.

Storing the checksum elsewhere may have prevented mutations on the checksum;
still, as demonstrated by the first case, such an effort is not itself a
solution. Hosting a checksum on a secondary system is not very useful if users
downloading the resource protected by that checksum cannot find it or are not
actually \emph{checking} it against a manual calculation. \\

\noindent\textbf{Case 3: Havex.} As part of a widespread espionage campaign
beginning in 2010, Russian Intelligence Services targeted the industrial control
systems of numerous aviation, national defense, critical infrastructure,
pharmaceutical, petrochemical, and other companies and organizations with the
Havex remote access trojan~\cite{SCA-HAVEX1, SCA-HAVEX2}. The attack was carried
out in phases whereby innocuous software images hosted on disparate
\emph{legitimate} vendor websites were targeted for replacement with versions
infected with the Havex malware~\cite{SCA-HAVEX2}. The goal here, as is the case
with all SCAs, was to infect victims indirectly by having the Havex malware
bundled into opaque software dependencies, \eg{ a hardware driver or internal
communication application}.

It is estimated that Havex successfully impacted hundreds or even thousands of
corporations and organizations---mostly in United States and
Europe~\cite{SCA-HAVEX2}. The motivation behind the Havex malware was
intelligence exfiltration and espionage~\cite{SCA-HAVEX1}. How many of these
vendors employed checksums and other mitigations as part of their software
release cycle is not well reported, though investigators note said vendors'
distribution mechanisms were insecure~\cite{SCA-HAVEX2}; however, an automated
resource verification method could have helped mitigate the delivery of
compromised software to users. \\

\noindent\textbf{Case 4: HandBrake.} In May of 2017, users of HandBrake, a
popular open source video transcoder for Mac/OSX, were made aware that they may
have downloaded and installed a trojan riding atop their transcoding software.
Attackers breached an HTTPS-protected HandBrake download mirror, replacing the
legitimate software with a version containing a novel variant of the
\emph{Proton} malware~\cite{SCA-HB1}. The number of users potentially affected
is unreported.

The goal of the attack was the exfiltration of victims' sensitive data,
including decrypted keychains, private keys, browser password databases,
1Password/Lastpass vaults, decrypted files, and victims' personal videos and
other media~\cite{SCA-HB1}. The HandBrake developers recommended users perform
manual checksum verification to determine if their installation media was
compromised~\cite{SCA-HB2}.

Despite the attackers mutating the HandBrake binary, the authoritative checksums
listed on the official HandBrake download page were reportedly left
untouched~\cite{SCA-HB2}. Further, the developers of HandBrake store their
authoritative checksums both on their official website and in their official
GitHub repository~\cite{SCA-HB2}. A sufficiently knowledgeable, sufficiently
motivated user \emph{might} have noticed the discrepancy between their
calculated checksum and the authoritative checksum listed on the download page.

Suppose, however, that the attackers \textit{had} managed to mutate the
checksums on the official website. Then there would be a discrepancy between the
``authoritative'' checksums on the official site and the ``authoritative''
checksums in the GitHub repository---that is \emph{if} users are even aware that
a second set of checksums are available at all. Which are the actual
authoritative checksums? On top of requiring the technical knowledge, a user in
this confusing situation is then expected to ``do the right thing,'' whatever
that happens to be in context.
