\section{Implementation} \label{sec:hc-implementation}

We implement a proof-of-concept HASCHK frontend as a Google Chrome extension.
To demonstrate the general applicability of our approach, our frontend currently
works with two backends: (1) directly with DNS via Google's JSON API for DNS
over HTTPS (DoH) and (2) indirectly with DHT (Ring OpenDHT) via a local
Representational State Transfer (REST) API we designed to mimic the response
syntax of Google's DoH JSON API. Other candidate backends include storage
clusters, relational and non-relational databases, and any high availability
key-value store.

Our extension operates following the high level overview in \figref{protocol}
with some key implementation-specific functionality that we explore in the
remainder of this subsection. After the user initiates a download either
directly (\eg{ typing a URI manually}) or by following a hyperlink, our
extension, having observed the initial web request, associates that request with
the new download item. The original request may have directly triggered the
download or the download may have been triggered after one or more redirections.
Our extension derives the correct BD in both cases (see below). At this point,
if the backend does not exist or does not conform to the protocol, the protocol
terminates. Otherwise, after the download completes, a URN is calculated and
combined with the BD to make a final query to the backend. In effect, this query
asks: \emph{is this resource compromised?} If the response to the query is
negative (\ie{ a DNS TXT record or DHT data value}), the extension considers the
resource validated. On the other hand, if the response to the query is positive
(\ie{ no record found or bad data value returned}), the extension considers the
resource compromised, deletes the resource file, and warns the user.

In keeping with UX suggestions of prior work, our extension is virtually
transparent to end users in the common case that a download is successfully
verified by a provider's backend or when HASCHK has not been properly
deployed by a provider. However, in the relatively uncommon case that a resource
is deemed compromised, the extension will delete the file and alert the user
with the primary option to ``dismiss'' the alert, mimicking Chrome's own highly
effective dangerous download warning~\cite{ChromeClickThrough}. Specifically:
the user has the option of clicking through the warning via a low-key secondary
interface where they can force the extension to ignore the compromised nature of
the resource \emph{the next time it is downloaded}, forcing the user to trigger
the download again. This inconvenience, favoring users' security over choice, is
in keeping with the recommendations of prior work regarding security warning UX
(see \secref{hc-motivation}).

As no JavaScript OpenDHT client implementation exists, we developed a REST API
to facilitate communication between our extension and the Ring OpenDHT network
via HTTP and JSON. In a production implementation, a JavaScript OpenDHT client
would be baked directly into the extension.

Deploying the DNS backend is as simple as adding certain TXT records to our DNS
zone. This is a straightforward operation. Similarly, deploying the DHT backend
requires adding certain key-value pairs to the OpenDHT network, which is
trivial. Moreover, both DNS and DHT backends are performant and highly
available. The Ring OpenDHT network is also free to use. In the case of DNS, the
vast majority of web-facing providers and IT teams are already using it, already
pay for their own DNS zones, and are already quite familiar with configuring and
managing them. Hence, no exotic secondary backend system is required for
providers with web servers. Additionally, we argue updating the URNs stored by
these backends automatically---by integrating a URN calculation and record
insertion/update step into a modern development toolchain or resource deployment
pipeline---is relatively low-effort and straightforward for providers.

No application or website source code changes, costly user-facing server or web
infrastructure customizations, or modifications to web standards are needed to
enable our extension to protect a provider's resources. Given a production-ready
implementation of our frontend---coupled with the near ubiquitous adoption of
DNS---HASCHK (with a DNS backend) could be deployed immediately by interested
providers.

\subsection{URNs, BDs, and Fallthrough in Google Chrome}

BD derivation in the browser is non-trivial. Users can initiate downloads
through clicking links inside \emph{and outside} the browser (\eg{ in an email
application}), through asynchronous JavaScript, and by entering URIs into the
browser manually. To catch these and other edge cases, we implement our
extension using the WebRequest API~\cite{ExtensionAPI}, which allows us to
monitor all navigation events, collate redirects into an interrogable chain of
requests, and associate new download items with the URI where they were
originally initiated.

To query the backend, we: (1) BASE32 encode the resource's URN. (2) Divide the
112-character result into two 56-character strings \texttt{C1} and \texttt{C2}.
(3) We calculate the BD, which is either the 3LD and 2LD from
\texttt{details.originUrl}. We choose by issuing up to two queries of the form
\texttt{AL.BD}, where the Application Label (AL) is the constant string
\texttt{\_haschk}, looking for a response containing the value \texttt{OK}. We
first query the 3LD as BD and, if we do not receive the proper response there,
query the 2LD as BD. If the proper response is not received from either query,
the extension assumes HASCHK has not been properly deployed and terminates.
(4) Otherwise, concatenating C1, C2, AL, and BD, we issue a query of the form
\texttt{C1.C2.AL.BD}, again looking for a response containing \texttt{OK}
(negative) or that the query was not found (positive).

To handle redirection, we use Chrome's Web Request API to associate redirected
requests with one another. The originating request's \texttt{origin} at the top
of this chain is used to derive the BD using the \texttt{details.originUrl} and
\texttt{details.documentUrl} properties (also avoiding iframe issues). This
prevents an adversary from trivially fooling HASCHK by replacing a legitimate
hyperlink with a malicious one. A naive implementation might use the
\texttt{DownloadItem.referrer} property to derive the BD, since it should be
populated with the originating URL. Unfortunately, this property can be
manipulated with one or more redirections causing the BD to point into the
attacker's DNS zone. If the attacker properly deploys a HASCHK backend, they
could then trivially force false negatives.

Keeping a chain of associated requests also allows us to implement
``fallthrough'' functionality where, if the original request at the head of the
chain has not deployed HASCHK, the extension can walk the chain, checking
each point of redirection for a proper HASCHK deployment. This ensures URL
shorteners and other redirection services do not trigger false positives.
